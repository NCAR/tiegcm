%
\section{Electric Field}
%
In \src{subroutine threed} the two and three dimensional 
electric field and the three dimensional electric potential 
is calculated from the two dimensional electric
potential $\Phi$. The components of the electric field $E_{d1}$ and $E_{d2}$
which are in more-or-less magnetic eastward and downward/equatorward direction are
determined by
%
\begin{align}
   E_{d1} &= - \frac{1}{R_0 cos \lambda_m}\frac{\partial \Phi}{\partial \phi_m} \\
   E_{d2} &= \frac{1}{R_0 sin I_m}\frac{\partial \Phi}{\partial \lambda_m}
\end{align}
%
In the code the equally spaced latitudinal grid point
distribution $\lambda_0$ in $\lambda_m^*$ is used to calculate the derivatives. Therefore 
the mapping factors from the irregular latitudinal spaced 
grid $\lambda_m^*$ to $\lambda_0$ have to be taken into account. 
Including these factors the discrete derivatives are
%
\begin{align}
   E_{d1}(i,j) &= - \frac{1}{R_0}\frac{cos \lambda_0(j)}{cos \lambda_m^*(j)}
           \frac{ \Phi(i-1,j) -\Phi(i+1,j) }{2 cos \lambda_0(j) \Delta \phi_m} \\
   E_{d2}(i,j) &= \frac{1}{R_0}\frac{\partial \lambda_0(j)}{|sin I_m(j)|
               \partial \lambda_m^*(j)}
            \frac{\Phi(i,j-1) -\Phi(i,j+1)}{2 \Delta \lambda_0}
\end{align}
%
The polar values are averaged over the four surrounding points. At the equator a
second order polynomial of the electric potential is fitted through the adjacent points
and thus the derivative of the polynomial at the equator is determined. \\

The three dimensional electric potential and electric field is calculated
assuming that the dipolar magnetic field lines are equipotential. At each grid 
point $(\phi(i),\lambda_m^*(j),h(k))$
the foot point of the magnetic field line going through this grid point is
determined. Having found the foot point of the field line at height $h_0$
the values, i.e. electric potential and electric field, 
at this point can be determined by a two dimensional interpolation of
the surrounding points. The polar values are determined by longitudinal 
averaging all the polar values.
