%
\section{Calculation of Minor species $N(^2D)$,$N(^4S)$, and $NO$  \index{COMP\_N2D.F} \index{COMP\_N4S.F} \index{COMP\_NO.F}}\label{cap:comp_n}
%
\subsection{Calculation of $N(^2D)$}\label{subcap:comp_n2d}
%
The input to \src{subroutine comp\_n2d} is summarized in table
\ref{tab:input_comp_n2d}.
%
\begin{table}[tb]
\begin{tabular}{|p{3.5cm} ||c|c|c|c|c|c|} \hline
physical field               & variable        & unit&pressure
level& timestep
\\ \hline \hline
%
mass mixing ratio $O_2$ &       $\Psi(O_2)$              & $-$   &  midpoints & $t_n$\\
mass mixing ratio $O$ &         $\Psi(O  )$              & $-$   &  midpoints & $t_n$\\
mass mixing ratio $NO$ &       $\Psi(NO)$              & $-$   &  midpoints & $t_n$\\
electron density&       $Ne$              & $1/cm^3$   &  interfaces & $t_n$\\
number density $O^+$&       $n(O^+)$              & $1/cm^3$   &  midpoints & $t_n$\\
number density $N(^2P)$&       $n(N(^2P))$              & $1/cm^3$   &  midpoints & $t_n$\\
number density $NO^+$&       $n(NO^+)$              & $1/cm^3$   &  midpoints & $t_n$\\
conversion factor&       $N \overline{m}$              & $\#/cm^3
g/mole$   &  midpoints & $t_n$
 \\ \hline
\end{tabular}
\caption{Input fields to \src{subroutine comp\_n2d}}
\label{tab:input_comp_n2d}
\end{table}
%
The output of \src{subroutine comp\_n2d} is summarized in table
\ref{tab:output_comp_n2d}.
%
\begin{table}[tb]
\begin{tabular}{|p{3.5cm} ||c|c|c|c|c|c|} \hline
physical field               & variable        & unit&pressure
level& timestep \\ \hline \hline mass mixing ratio $N(^2D)$ &
$\Psi(N(^2D))$ & $-$ & midpoints & $t_n+\Delta t$
\\ \hline \hline
\end{tabular}
\caption{Output fields of \src{subroutine comp\_n2d}}
\label{tab:output_comp_n2d}
\end{table}
%
$N(^2D)$ is assumed to be in photochemical equilibrium, such that
the production $P(N(^2D))$ and loss rate $L(N(^2D))$ of $N(^2D)$ is
balanced.
%
\begin{align}
  n(N(^2D)) = \frac{P(N(^2D))}{L(N(^2D))}
\end{align}
%
The production is due to solar EUV photodissociation of $N_2$, and
ion and neutral chemistry
%
\begin{align}
 P(N(^2D))&(z+\frac{1}{2}\Delta z) = \frac{1}{2} \left[ Q_{tef}(z) +  Q_{tef}(z+\Delta z)\right]
   br_{N^2D} + \notag \\
   & k_3 n(N_2^+)N \overline{m}\frac{\Psi(O)}{m_o} +
   ( \alpha_1 n(NO^+) 0.85 + \notag \\
   & \alpha_3 n(N_2^+) 0.9 )
   \sqrt{Ne(z) Ne(z+\Delta z}
\end{align}
%
with the branching ratio of producing $N(^2D)$ which is set to 0.6.
The chemical reaction rates are $\alpha$ and $k$ defined in the
\src{chemrates\_module}. The total production rate due to solar
radiation and auroral particle precipitation is denoted by $Q_{tef}$
which is calculated in \src{subroutine qrj}. \\
The loss term $L(N(^2D))$ is determined by
%
\begin{align}
  L(N(^2D))(z+\frac{1}{2}\Delta z) = N\overline{m} \left[ \beta_2 \frac{\Psi(O_2)}{m_{O_2}} +
     \beta_4 \frac{\Psi(O)}{m_{O}} +  \beta_6
     \frac{\Psi(NO)}{m_{NO}}\right]+ \notag \\
     \; \; \; \beta_7 + \beta_5\sqrt{Ne(z) Ne(z+\Delta z)} + k_{10}n(O^+)
\end{align}
%
The chemical rates $\beta$ are defined in the module
\src{chemrates\_module}. \\
The number density $n(N(^2D))$ is converted to the mass mixing ratio
by
%
\begin{align}
  \Psi(N(^2D) = \frac{m_{N^2D}}{N \overline{m}}\frac{P(N(^2D))}{L(N(^2D))}
\end{align}
%
\subsection{Calculation of $N(^4S)$}\label{subcap:comp_n4s}
%
The input to \src{subroutine comp\_n4s} is summarized in table
\ref{tab:input_comp_n4s}.
%
\begin{table}[tb]
\begin{tabular}{|p{3.5cm} ||c|c|c|c|c|c|} \hline
physical field               & variable        & unit&pressure
level& timestep
\\ \hline \hline
%
mass mixing ratio $O_2$ &       $\Psi(O_2)$              & $-$   &  midpoints & $t_n$\\
mass mixing ratio $O$ &         $\Psi(O  )$              & $-$   &  midpoints & $t_n$\\
mass mixing ratio $NO$ &       $\Psi(NO)$              & $-$   &  midpoints & $t_n$\\
mass mixing ratio $N(^2D)$ &       $\Psi(N(^2(D))$              & $-$   &  midpoints & $t_n$\\
electron density&       $Ne$              & $1/cm^3$   &  interfaces & $t_n$\\
number density $O_2^+$&       $n(O_2^+)$              & $1/cm^3$   &  midpoints & $t_n$\\
number density $O^+$&       $n(O^+)$              & $1/cm^3$   &  midpoints & $t_n$\\
number density $N_2^+$&       $n(N_2^+)$              & $1/cm^3$   &  midpoints & $t_n$\\
number density $N^4$&       $n(N^4)$              & $1/cm^3$   &  midpoints & $t_n$\\
number density $NO^+$&       $n(NO^+)$              & $1/cm^3$   &  midpoints & $t_n$\\
neutral temperature&       $Tn$              & $K$   &  midpoints & $t_n$\\
mean molecular mass&       $\overline{m}$              & $g/mole????$   &  midpoints & $t_n$\\
conversion factor&       $N \overline{m}$              & $\#/cm^3
g/mole$   &  midpoints & $t_n$
 \\ \hline
\end{tabular}
\caption{Input fields to \src{subroutine comp\_n4s}}
\label{tab:input_comp_n4s}
\end{table}
%
The output of \src{subroutine comp\_n4s} are the factor which define
the lower boundary calculated in \src{subroutine minor} by solving
$A\frac{d X}{dZ} + B X + C = 0$. Also the upward flux at the upper
boundary is defined, and the production $P$ and loss rate $L$ of
$N(^4S)$. These values are stored in module data of the module
\src{comp\_n4s}. After \src{subroutine comp\_n4s} the
\src{subroutine minor\_n4s} is called to calculate the mass mixing
ratio of $N(^4S)$. \src{Subroutine minor\_n4s} is not described here
since it just calls \src{subroutine minor}. For information about
\src{subroutine minor} please see chapter \ref{cap:minor}. The input
to \src{subroutine minor} is described in table
\ref{tab:input_minor} and the output in table
\ref{tab:output_minor}, subsituting the species $X$ by $N(^4S)$. \\

%
%
$N(^4S)$ has longer life time and is therefore influenced by the
neutral winds. \\
The lower boundary is given by photochemical equilibrium. The
factors $A$, $B$, and $C$ for the equation solved in \src{subroutine
minor} $A\frac{d \Psi(N(^4S))}{dZ} + B \Psi(N(^4S)) + C = 0$ are
%
\begin{align}
  A =& 0 \notag \\
  B =& 1. \notag \\
  C =& (N(^4S))= \frac{P(N(^4S))}{L(N(^4S))}
\end{align}
%
The conversion factor from number density to mass mixing ratio $N
\overline{m}(z_{LB})$ at the lower boundary is
%
\begin{align}
 N \overline{m}(z_{LB})= p_0 e^{-z_{bot}-1/2\Delta z} e^{1/2\Delta
 z}\frac{\overline{m}(z_{bot})}{k_B T_n(z_{bot})}
\end{align}
%
The mass mixing ratio ????? or number density at the lower boundary
is then
%
\begin{align}
 C = & n(N(^4S))(z_{lbc})= \frac{m_{n(^4S)}}{N\overline{m}}[ Q_{tef}(1-br_{N^2D})
  \frac{1}{N\overline{m}(z_{bot})} + \notag \\
  & \frac{\Psi(N(^2D))}{m_{N^2D}}(
  \beta_4 N \overline{m}(z_bot)\frac{\Psi(NO)}{m_o}+ \beta_5 Ne + \beta_7)
  + \beta_8\frac{\Psi(NO)}{m_{NO}}]\notag \\
   & \frac{1}{\beta_1
  \frac{\Psi(O_2)}{m_{O_2}}+\beta_3\frac{\Psi(NO}{m_{NO}}+N\overline{m} \beta_{17}
  \frac{\Psi(O)}{m(O)}\frac{\Psi(N_2)}{m_{N_2}}}
\end{align}
%
At the upper boundary the upward diffusive flux  is set
%
\begin{align}
 n(N(^4S))(z_{ubc})= 0.
\end{align}
%
The production rate $P$ of $N(^4S)$ is
%
\begin{align}
P(N(^4S))=& \frac{1}{2} (Q_{tef}(z)+Q_{tef}(z+\Delta
z))(1-br_{N^2D})
+ \notag \\
& N\overline{m} [\frac{\Psi(N^2D)}{m_{N^2D}}\left\{ \beta_5
 \frac{1}{2}(Ne(z) + Ne(z+\Delta z)+\beta_7\right\} + \notag \\
&  \frac{1}{2}(\beta_8(z) + \beta_8(z+\Delta
 z))\frac{\Psi(NO)}{m_{NO}}]+  \notag \\
& N\overline{m} [ k_2 n(O^+)\frac{\Psi(xN_2)}{m_{N_2}}+ k_6
 n(N^+)\frac{\Psi(O_2)}{m_{O_2}} k_8
 n(N^+)\frac{\Psi(O)}{m_{O}}+ \notag \\
& \sqrt{Ne(z) \cdot Ne(z+\Delta z}
 (\alpha_1 n(NO^+)0.15 + \alpha_3 n(N_2^+)1.1)]
\end{align}
%
and the loss rate $L$ of $N(^4S)$ is
%
\begin{align}
  L(N(^4S)) = & -N\overline{m} \left[ \beta_1 \frac{\Psi(O_2)}{m_{O_2}} +
   \beta_3 \frac{\Psi(NO)}{m_{NO}} + N \overline{m} \beta_{17} \frac{\Psi(O)}{m_{O}}
    \frac{\Psi(xN_2)}{m_{N_2}} \right] - \notag \\
    &  k_4 n(O_2^+)
\end{align}
%
%
%
\subsection{Calculation of $NO$}\label{subcap:comp_no}
%
The input to \src{subroutine comp\_no} is summarized in table
\ref{tab:input_comp_no}.
%
\begin{table}[tb]
\begin{tabular}{|p{3.5cm} ||c|c|c|c|c|c|} \hline
physical field               & variable        & unit&pressure
level& timestep
\\ \hline \hline
%
mass mixing ratio $O_2$ &       $\Psi(O_2)$              & $-$   &  midpoints & $t_n$\\
mass mixing ratio $O$ &         $\Psi(O  )$              & $-$   &  midpoints & $t_n$\\
mass mixing ratio $N(^4S)$ &       $\Psi(N(^4S))$              & $-$   &  midpoints & $t_n$\\
mass mixing ratio $N(^2D)$ &       $\Psi(N(^2(D))$              & $-$   &  midpoints & $t_n$\\
electron density&       $Ne$              & $1/cm^3$   &  interfaces & $t_n$\\
number density $O_2^+$&       $n(O_2^+)$              & $1/cm^3$   &  midpoints & $t_n$\\
neutral temperature&       $Tn$              & $K$   &  midpoints & $t_n$\\
mean molecular mass&       $\overline{m}$              & $g/mole????$   &  midpoints & $t_n$\\
conversion factor&       $N \overline{m}$              & $\#/cm^3
g/mole$   &  midpoints & $t_n$
 \\ \hline
\end{tabular}
\caption{Input fields to \src{subroutine comp\_no}}
\label{tab:input_comp_no}
\end{table}
%
The output of \src{subroutine comp\_no} are the factors $A$, $B$,
and $C$ which define the lower boundary condition of  $NO$ , the
upper boundary upward diffusive flux, and the production $P$ and
loss rate $L$ of $NO$. These values are stored in module data of the
module \src{comp\_no}. After \src{subroutine comp\_no} the
\src{subroutine minor\_no} is called to calculate the mass mixing
ratio of $NO$. \src{Subroutine minor\_no} is not described here
since it just calls \src{subroutine minor}. For information about
\src{subroutine minor} please see chapter \ref{cap:minor}. The input
to \src{subroutine minor} is described in table
\ref{tab:input_minor} and the output in table
\ref{tab:output_minor}, subsituting the species $X$ by $NO$. \\


%
$NO$ has longer life time and is therefore influenced by the neutral
winds. The lower boundary is given by photochemical equilibrium. The
factors $A$, $B$, and $C$ for the equation solved in \src{subroutine
minor} $A\frac{d \Psi(N(^4S))}{dZ} + B \Psi(N(^4S)) + C = 0$ are
%
\begin{align}
  A =& 0 \notag \\
  B =& 1. \notag \\
  C =&??? \Psi(NO)= -\frac{n(NO)(z_{LB}) m_{NO} k_B T_n(z_{LBC})}
  {p_0 e^{1/2 \Delta z} e^{-z -1/2 \Delta z}\overline{m}(z_{LBC}}
\end{align}
%
with $n(NO)(z_{LB}) = 4 \cdot 10^6$. At the upward diffusive flux at
the upper boundary is set to zero
%
\begin{align}
 n(N(^4S))(z_{ubc})= 0.
\end{align}
%
The production rate $P$ of $NO$ is
%
\begin{align}
 P(NO) = & (N\overline{m})^2 \frac{\Psi(O_2)}{m_{O_2}} \left(
    + \beta_2 \frac{\Psi(N(^2D))}{m_{N^2D}} \right) + \notag \\
  & (N\overline{m})^3 \beta_{17}\frac{\Psi(N_2)}{m_{N_2}}
   \frac{\Psi(N(^4S))}{m_{N^4S}}
\end{align}
%
and the loss rate $L$ of $NO$ is
%
\begin{align}
 L(NO) = & -N\overline{m} \left[ \beta_3 \frac{\Psi(N(^4S))}{m_{N(^4S)}}
   +  \beta_6 \frac{\Psi(N(^2D))}{m_{N(^2D)}}\right] -\frac{1}{2}
   \left( \beta_8(z) + \beta_8(z+\Delta z)  \right) + \notag \\
   & \frac{1}{2}
   \left( \beta_9(z) + \beta_9(z+\Delta z)  \right) - k_9 n(O_2^+)
   \frac{1}{2}
   \left( \beta_{9n}(z) + \beta_{9n}(z+\Delta z)  \right)
\end{align}
%
