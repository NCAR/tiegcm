%
\section{Calculation of Major species $O_2$, and $O$  \index{COMP\_O2O.F}, \index{comp.F}}\label{cap:comp_o2o}
%
\subsection{Calculation of source and sinks for $O_2$ and $O$}\label{subcap:comp_o2o}
%
The input to \src{subroutine comp\_O2O} is summarized in table
\ref{tab:input_comp_o2o}.
%
\begin{table}[tb]
\begin{tabular}{|p{3.5cm} ||c|c|c|c|c|c|} \hline
physical field               & variable        & unit&pressure
level& timestep
\\ \hline \hline
%
mass mixing ratio $O_2$ &       $\Psi(O_2)$              & $-$   &  midpoints & $t_n$\\
mass mixing ratio $O$ &         $\Psi(O  )$              & $-$   &  midpoints & $t_n$\\
electron density&       $Ne$              & $1/cm^3$   &  interfaces & $t_n$\\
number density $O^+$&       $n(O^+)$              & $1/cm^3$   &  midpoints & $t_n$\\
number density $N_2^+$&       $n(N_2^+)$              & $1/cm^3$   &  midpoints & $t_n$\\
number density $NO^+$&       $n(NO^+)$              & $1/cm^3$   &  midpoints & $t_n$\\
number density $NO$&       $n(NO)$              & $1/cm^3$   &  midpoints & $t_n$\\
number density $N(^4S)$&       $n(N(^4S))$              & $1/cm^3$   &  midpoints & $t_n$\\
number density $N^+$&       $n(N^+)$              & $1/cm^3$   &  midpoints & $t_n$\\

 \\ \hline
\end{tabular}
\caption{Input fields to \src{subroutine comp\_o2o}}
\label{tab:input_comp_o2o}
\end{table}
%
The output of \src{subroutine comp\_o2o} is summarized in table
\ref{tab:output_comp_o2o}.
%
\begin{table}[tb]
\begin{tabular}{|p{3.5cm} ||c|c|c|c|c|c|} \hline
physical field               & variable        & unit&pressure
level& timestep \\ \hline \hline matrix coefficients for rhs? & $fs$
& $-$ & midpoints? & $t_n+\Delta t$
\\ \hline \hline
\end{tabular}
\caption{Output fields of \src{subroutine comp\_o2o}}
\label{tab:output_comp_o2o}
\end{table}
%
The module data of \src{subroutine comp\_o2o} is summarized in table
\ref{tab:module_comp_o2o}.
%
\begin{table}[tb]
\begin{tabular}{|p{3.5cm} ||c|c|c|c|c|c|} \hline
physical field               & variable        & unit&module&
timestep
\\ \hline \hline inverse of molecular mass $O_2$&
$1/m_{O_2}$ & $mole/g$ & cons & $-$ \\
inverse of molecular mass $O$& $1/m_{O}$ & $mole/g$ & cons & $-$ \\
inverse of molecular mass $N_2$& $1/m_{N_2}$ & $mole/g$ & cons & $-$ \\
reference pressure & $p_0$ & $???$ & cons & $-$ \\
- & $e^z$ & $-$ & cons & $-$ \\
Boltzman constant & $k_B$ & $-$ & cons & $-$ \\
 molecular mass $NO$& $m_{NO}$ & $g/mole$ & cons & $-$ \\
 molecular mass $N(^4S)$& $m_{N^4S}$ & $g/mole$ & cons & $-$ \\
 molecular mass $N(^2D)$& $m_{N^2D}$ & $g/mole$ & cons & $-$ \\
 molecular mass $O_2$& $m_{O_2}$ & $g/mole$ & cons & $-$ \\
 molecular mass $O$& $m_{O}$ & $g/mole$ & cons & $-$ \\
 total $O_2$ dissociation frequency& $rj$ & $1/s$ & qrj & $interface$ \\
 $O_2^+$ production rate& $Q(O_2^+)$ & $??$ & qrj & $interface$ \\
 $O^+$ production rate& $Q(O^+)$ & $??$ & qrj & $interface$ \\
 chemical reaction rates& $k_i$ & $???$ & chemrates & $-$ \\
 chemical reaction rates& $\beta_i$ & $???$ & chemrates & $-$ \\
 chemical reaction rates& $a_i$ & $???$ & chemrates & $-$ \\
 coefficient matrix & $fs$ & $???$ & chemrates & $-$
\\ \hline \hline
\end{tabular}
\caption{Module data used in \src{subroutine comp\_o2o}}
\label{tab:module_comp_o2o}
\end{table}
%
The source and sink terms for the neutral composition are calculated
in this subroutine, and stored in the coefficient matrix $fs$, which
is module data of \src{chemrates\_module}. \\

Firstly, the total dissociation frequency and the production rates
are transferred from the interface level to the midpoint level.
%
\begin{align}
  rj(z+\frac{1}{2}\Delta z ) = \frac{1}{2} [rj(z)+ rj(z+\Delta z)] \notag \\
  Q(O_2^+)(z+\frac{1}{2}\Delta z ) = \frac{1}{2} [Q(O_2^+)(z)+ Q(O_2^+)(z+\Delta z)] \notag \\
  Q(O^+)(z+\frac{1}{2}\Delta z ) = \frac{1}{2} [Q(O^+)(z)+ Q(O^+)(z+\Delta z)]
\end{align}
%
The conversion factor $N\overline{m}$ from number density to mass
mixing ratio is evaluated at the midpoint level
%
\begin{align}
  N\overline{m}(z+\frac{1}{2}\Delta z) = \frac{p_0 e^{-z-1/2\Delta z}}{k_B T_n(z+1/2\Delta
  z)}\left( \frac{\Psi(O_2)}{m_{O_2}}+ \frac{\Psi(O)}{m_{O}}+\frac{\Psi(N_2)}{m_{N_2}}\right)
\end{align}
%
with $\Psi(N_2) = 1- \Psi(O_2 )- \Psi(O)$, and
$1/\overline{m}(z+\frac{1}{2}\Delta z) =\frac{\Psi(O_2)}{m_{O_2}}+
\frac{\Psi(O)}{m_{O}}+\frac{\Psi(N_2)}{m_{N_2}} $. The $O_x$
production is
%
\begin{align}
  P(O_x) = & (N\overline{m})^2 \left[ \beta_3 \frac{n(N^4S)}{m_{N^4S}}\frac{n(NO)}{m_{NO}}
     + \beta_6 \frac{n(N^2D)}{m_{N^2D}}\frac{n(NO)}{m_{NO}}\right] +
     \notag \\
  & \frac{1}{2}( \beta_8(z) + \beta_8(z+\Delta
    z))\frac{n(NO)}{m_{NO}}N\overline{m}(z+\frac{1}{2}\Delta z) + \notag \\
  & N\overline{m}(z+\frac{1}{2}\Delta z) \left[ k_4 n(O_2^+)\frac{n(N^4S)}{m_{N^4S}} +
     k_4 n(O^+)\frac{n(N^2D)}{m_{N^2D}}\right] + \notag \\
  & \sqrt{Ne(z)Ne(z+\Delta z)}(a_1 n(NO^+)= 2 a_2 n(O_2^+))
\end{align}
%
???? what is mass mixing ratio and what number density ??? and
%
\begin{align}
 P(O_{x2}) = & (N\overline{m})(z+\frac{1}{2}\Delta z)\left[ \beta_1 \frac{n(N^4S)}{m_{N^4S}}
    + \beta_2 \frac{n(N^2D)}{m_{N^2D}}\right] + \notag \\
    & k_1 n(O^+) + k_7 n(N^+) + 2 rj(z+\frac{1}{2} \Delta z)
\end{align}
%
The $O_x$ loss rates are
%
\begin{align}
   L(O_{x1}) = & 2 k_{m12} N\overline{m} \frac{1}{\frac{1}{2}(\overline{m}(z) + \overline{m}(z+\Delta
   z))} \\
   L(O_{x2}) = & k_3 n(N_2^+) + k_8 n(N^+) \\
   L(O_{x3}) = & Q(O^+) (z+\frac{1}{2} \Delta z)
\end{align}
%
The $O_2$ production rates are
%
\begin{align}
  P(O_2)_1 = & k_{m12}\frac{N\overline{m}(z+\frac{1}{2} \Delta z)}{\frac{1}{2}(\overline{m}(z) + \overline{m}(z+\Delta
   z))} \\
  P(O_2)_2 = & 0 \\
  P(O_2)_3 = & k_5 \frac{n(NO}{m_{NO}}n(O_2^+)N\overline{m}(z+\frac{1}{2} \Delta z)
\end{align}
%
The $O_2$ loss rates are
%
\begin{align}
  L(O_2)_1 = & N\overline{m}(z+\frac{1}{2} \Delta z)\left[ \beta_1 \frac{n(N^4S)}{m_{N^4S}} +
      \beta_2 \frac{n(N^2D)}{m_{N^2D}}\right] + \notag \\
      & k_1 n(O^+) + (k_6 + k_7)n(N^+) + k_9 n(N_2^+) + rj \\
  l(O_2)_2 = Q(O_2^+)(z+\frac{1}{2} \Delta z)
\end{align}
%
The coefficients of the matrix $\mathbf(fs)$ are then
%
\begin{align}
  fs_{11} = & - L(O_2) \\
  fs_{12} = & N\overline{m}(z+\frac{1}{2} \Delta z) P(O_2)_1
        \frac{\Psi(O)}{m_O}\frac{m_{O_2}}{m_O} \\
  fs_{21} = & P(O_x)_2 \frac{m_O}{}m_{O_2} \\
  fs_{22} = & -L(O_x)_2 - L(O_x)_1\frac{\Psi(O)}{m_O}N\overline{m}(z+\frac{1}{2} \Delta
         z)\\
  fs_{10} = & ( P(O_2)_3 - L(O_2)_2) \frac{m_{O_2}}{N\overline{m}} \\
  fs_{20} = & ( P(O_x)_1 - L(O_x)_3) \frac{m_{O}}{N\overline{m}} \\
\end{align}
%
%
\subsection{Calculation of major species  $O_2$ and $O$}\label{subcap:comp}
%
The input to \src{subroutine comp} is summarized in table
\ref{tab:input_comp}.
%
\begin{table}[tb]
\begin{tabular}{|p{3.5cm} ||c|c|c|c|c|c|} \hline
physical field               & variable        & unit&pressure
level& timestep
\\ \hline \hline
%
mass mixing ratio $O_2$ &       $\Psi(O_2)$              & $-$   &  midpoints & $t_n$\\
mass mixing ratio $O$ &         $\Psi(O  )$              & $-$   &  midpoints & $t_n$\\
mass mixing ratio $O_2$ &       $\Psi(O_2)$              & $-$   &  midpoints & $t_n-\Delta t$\\
mass mixing ratio $O$ &         $\Psi(O  )$              & $-$   &  midpoints & $t_n- \Delta t$\\
neutral temperature &        $T_n$              & $K$   &  midpoints & $t_n$\\
neutral zonal velocity &       $u_n$              & $cm/s$   &  midpoints & $t_n$\\
neutral meridional velocity &       $v_n$              & $cm/s$   &  midpoints & $t_n$\\
neutral "vertical" velocity &       $W$              & $1/s$   &  interfaces & $t_n$\\
horizontal diffusion of $O_2$ &       $h_a(O_2)$              & $??$   &  midpoints ??? & $t_n$\\
horizontal diffusion of $O$ &       $h_d(O)$              & $??$ &
midpoints ??? & $t_n$
 \\ \hline
\end{tabular}
\caption{Input fields to \src{subroutine comp}}
\label{tab:input_comp}
\end{table}
%
The output of \src{subroutine comp} is summarized in table
\ref{tab:output_comp}.
%
\begin{table}[tb]
\begin{tabular}{|p{3.5cm} ||c|c|c|c|c|c|} \hline
physical field               & variable        & unit&pressure
level& timestep \\ \hline \hline
updated mass mixing ratio $O_2$ &       $\Psi(O_2)^{upd}$              & $-$   &  midpoints & $t_n$\\
updated mass mixing ratio $O$ &         $\Psi(O  )^{upd}$              & $-$   &  midpoints & $t_n$\\
updated mass mixing ratio $O_2$ &       $\Psi(O_2)^{upd}$              & $-$   &  midpoints & $t_n+\Delta t$\\
updated mass mixing ratio $O$ &         $\Psi(O  )^{upd}$ & $-$   &
midpoints & $t_n+ \Delta t$
\\ \hline \hline
\end{tabular}
\caption{Output fields of \src{subroutine comp}}
\label{tab:output_comp}
\end{table}
%
The module data of \src{subroutine comp} is summarized in table
\ref{tab:module_comp}.
%
\begin{table}[tb]
\begin{tabular}{|p{3.5cm} ||c|c|c|c|c|c|} \hline
physical field               & variable        & unit&module&
timestep
\\ \hline \hline inverse of molecular mass $O_2$&
$1/m_{O_2}$ & $mole/g$ & cons & $-$ \\
inverse of molecular mass $O$& $1/m_{O}$ & $mole/g$ & cons & $-$ \\
inverse of molecular mass $N_2$& $1/m_{N_2}$ & $mole/g$ & cons & $-$ \\
- & $e^{-z-\frac{1}{2}\Delta z}$ & $-$ & cons & $-$ \\
- & $e^{-\frac{1}{2}\Delta z}$ & $-$ & cons & $-$ \\
- & $e^{\frac{1}{2}\Delta z}$ & $-$ & cons & $-$ \\
 molecular mass $O_2$& $m_{O_2}$ & $g/mole$ & cons & $-$ \\
 molecular mass $O$& $m_{O}$ & $g/mole$ & cons & $-$ \\
 & $\Pi$ & $-$ & cons & $-$ \\
 & $\frac{1}{2\Delta t}$ & $$ & cons & $-$ \\
smoothing factor & $f_{smo}=0.95$ & $$ & cons & $-$ \\
 & $\frac{1}{2}(1-0.95)$ & $$ & cons & $-$ \\
background eddy diffusion & $d_k$ & $$ & cons & $-$ \\
wave filtering & $k_{ut5}$ & $$ & cons & $-$ \\
matrix coefficients with source and sinks & $\mathbf{fs}$ & $$ &
chemrates & $-$
\\ \hline \hline
\end{tabular}
\caption{Module data used in \src{subroutine comp}}
\label{tab:module_comp}
\end{table}
%
The calculation of the composition is explained in
\cite{dickinson1984}
% Dickinson, R.E., Ridley, E.C., Roble, R.G.:
% Thermospheric General Circulation with Coupled Dynamics and Composition
% Journal of the atmospheric sciences, Vol 41, No. 2,pp 205-219
The major neutral constituents of the thermosphere are $O_2$, $O$,
and $N_2$. The mass mixing ratio of these are
%
\begin{align}
 \Psi_i = n_i m_i \left( \sum_{j=1}^3 n_j m_j \right)^{-1}
\end{align}
%
with $n_i$ the number density of the $i th$ species, with $i=1,2,
\text{and} 3$ corresponding to $O_2$, $O$, and $N_2$. The molecular
mass of the $i th$ species is $m_i$. The vector of mass mixing
ratios is defined by
%
\begin{align}
 \mathbf{\Psi} = \left(
                    \begin{array}{c}
                      \Psi_{O_2} \\
                      \Psi_O \\
                    \end{array}
                  \right)
\end{align}
%
The $N_2$ mass mixing ratio is determined by
%
\begin{align}
 \Psi_{N_2} = 1 - \Psi_{O_2}-\Psi_O
\end{align}
%
The coupled vector equation for the mass mixing ratio vector
\cite{dickinson1984}
%
\begin{align}
 \frac{\partial}{\partial t}\mathbf{\Psi} & = -e^z \tau^{-1}\frac{\partial}{\partial z}
 \left[ \frac{\overline{m}}{m_{N_2}\left(\frac{T_{00}}{T_n}\right)^{0.25}}
 \mathbf{\alpha}^{-1} \mathbf{L} \mathbf{\Psi}\right] + \notag \\
 & e^z \frac{\partial}{\partial z} \left( K(z)e^{-z}\frac{\partial}{\partial z}\mathbf{\Psi}\right)
 - \notag \\
 & \left( \mathbf{v}_n \cdot \nabla \mathbf{\Psi} +
 W \frac{\partial}{\partial z} \mathbf{\Psi}\right)+ \mathbf{S} -
 \mathbf{R}
\end{align}
%
with time $t$, the vertical coordinate $z= ln(p_0/p)$, the pressure
$p$ and $p_0$ the reference pressure. The diffusion time scale is
$\tau = (p_0 H_0/ p_{00}D_0) = 1.86 10^3 s$, with $H_0$ the
characteristic molecular nitrogen scale height at $T_{00}=273 K$,
$p_{00}= 10^5 Pa$, the characteristic diffusion coefficient at
$p_{00}$ and temperature $T_{00}$ is $D_0 = 2 10^{-5} m^2/s$. The
mean molecular mass is $\overline{m}$, the molecular mass of each
species is $m_i$. The horizontal neutral velocity is $\mathbf{v}_n$,
and the "vertical" velocity is $W=\frac{dz}{dt}$. $K(z)$ is the eddy
diffusion coefficient, the effective mass source is $\mathbf{S}$,
and the removal rate is $\mathbf{R}$. \\
The matrix $\mathbf{\alpha}$ varies as the inverse of the diffusion
coefficients according to
%
\begin{align}
  \alpha_{11} = & -[\phi_{13}+(\phi_{12}-\phi_{13})\Phi_{2}] \notag \\
  \alpha_{22} = & -[\phi_{23}+(\phi_{21}-\phi_{23})\Phi_{1}] \notag \\
  \alpha_{12} = &  (\phi_{12}-\phi_{13})\Phi_{1} \notag \\
  \alpha_{21} = &  (\phi_{12}-\phi_{13})\Phi_{2} \label{eq:comp_phi_ij}
\end{align}
%
where $i=1, 2, \text{and} 3$ denotes $O_2$, $O$, and $N_2$
respectively. The value $\phi_{ij}$ is determined by
%
\begin{align}
  \phi_{ij} = \frac{D}{D_{ij}}\frac{m_3}{m_j}
\end{align}
%
with $D_{ij}$ is the mutual diffusion coefficient for gases $i$ and
$j$ with $D_{12}$, $D_{13}$, $D_{23}$, and $D_0$ being $0.26, \;
0.18, \; 0.26$ \cite{colegrove1966}, and $0.2 \times 10^{-4} m^2/s$,
respectively.
% colegrove, F.D. 1966: Atmospheric composition in the lower thermosphere.
% J. Geophys. Res. 71, 2227-2236.
%
$D$ is the characteristic diffusion coefficient, which has the same
temperature and pressure dependence as $D_{ij}$
%
\begin{align}
  D = D_0 \frac{P_{00}}{P}\left(\frac{P}{T_{00}} \right)^{1.75}
\end{align}
%
The matrix operator $\mathbf{L}$ is diagonal with elements
%
\begin{align}
  \mathbf{L}= \delta_{ij}\left( \frac{\delta}{\delta z} - \epsilon_{ii}\right)
\end{align}
%
and
%
\begin{align}
  \epsilon_{ii} = 1-\frac{m_i}{\overline{m}}-
  \frac{1}{\overline{m}}\frac{\partial}{\partial z}
\end{align}
%
The matrix $\mathbf{L}$ defines diffusive equilibrium through
$\mathbf{L}\mathbf{\Phi} = 0$. at the upper boundary diffusive
equilibrium is assumed, so that $\mathbf{L}\mathbf{\Phi} = 0$. At
the lower boundary the atomic oxygen concentration $n(O) $ peaks,
which was observed, and therefore it is assumed $\partial
n(O)/\partial z = 0$. Together with the the conservation of total
oxygen atoms, this assumption leads to the following condition at
the lower boundary
%
\begin{align}
   \frac{\partial}{\partial z}\Psi_2 & = \Psi_2 \notag \\
   \Psi_1 + \Psi_2 & = constant
\end{align}
%

In the source code in \src{comp.F} first the values $\phi_{ij}$ from
equation (\ref{eq:comp_phi_ij}) are set
%
\begin{align}
  \phi_{21} = & \frac{0.2}{0.26}\frac{28}{32} = & 0.673 \notag \\
  \phi_{12} = & \frac{0.2}{0.26}\frac{28}{10} = & 1.35  \notag \\
  \phi_{13} = & \frac{0.2}{0.18}\frac{28}{28} = & 1.11  \notag \\
  \phi_{23} = & \frac{0.2}{0.26}\frac{28}{28} = & 0.769
\end{align}
%
The diffusive time scale $\tau$ is set to $1.86 \times 10^3 s$, and
the standard temperature $T_{00}= 273 K$. A $\mathbf{\delta}$ matrix
is set up as the identity matrix
%
\begin{align}
 \mathbf{\delta} = \left(
   \begin{array}{cc}
    1 & 0 \\
    0 & 1 \\
  \end{array}
 \right)
\end{align}
%
The horizontal advection term $\mathbf{v}_n \cdot \nabla
\mathbf{\Psi}$ is determined in the \src{subroutine advecl} which is
part of the file \src{comp.F}. The input to the \src{subroutine
advecl} is the mass mixing ratios $\Psi(O_2)^{t_n}$ and
$\Psi(O)^{t_n}$, as well as the horizontal neutral velocities $u_n$
and $v_n$.
%
\begin{align}
 & \mathbf{v}_n \cdot \Psi_i  = \frac{1}{2} \frac{1}{R_e \cos \lambda}
  [ \frac{2}{3 \Delta \phi} \left( \Psi_i(\phi + \Delta \phi, \lambda) -
   \Psi_i(\phi - \Delta \phi, \lambda) \right) \notag \\
   & \left( u_n(\phi + \Delta \phi, \lambda)
   +  u_n(\phi - \Delta \phi, \lambda)\right) - \notag \\
   & \frac{1}{12 \Delta \phi} \left( \Psi_i(\phi + 2\Delta \phi, \lambda) -
   \Psi_i(\phi - 2\Delta \phi, \lambda) \right) \left( u_n(\phi + 2\Delta \phi, \lambda)
   +  u_n(\phi - 2\Delta \phi, \lambda)\right)] + \notag \\
   & \frac{1}{2}\frac{1}{R_e}
  [ \frac{2}{3 \Delta \lambda} \left( \Psi_i(\phi, \lambda+ \Delta \lambda) -
   \Psi_i(\phi , \lambda-\Delta \lambda) \right) \left( v_n(\phi , \lambda+\Delta \lambda)
   +  v_n(\phi , \lambda-\Delta \lambda)\right) - \notag \\
   & \frac{1}{12 \Delta \lambda} \left( \Psi_i(\phi , \lambda+2\Delta \lambda) -
   \Psi_i(\phi , \lambda-2\Delta \lambda) \right) \left( v_n(\phi , \lambda\Delta \lambda)
   +  v_n(\phi , \lambda-2 \Delta \lambda)\right)]
\end{align}
%
for $i = 1, 2$ which corresponds to $O_2$ and $O$, respectively. The
mass mixing ratios at timestep $t_n - \Delta t$ are smoothed with a
two part Shapiro smoother. The smoothing in latitude is
%
\begin{align}
  \Psi_i^{t-\Delta t, \lambda_s}& = \Psi_i^{t-\Delta t} - 0.03 \{\Psi_i^{t-\Delta t}(\phi,\lambda +
  2 \Delta \lambda) + \Psi_i^{t-\Delta t}(\phi,\lambda -
  2 \Delta \lambda) - \notag \\
  & 4 [\Psi_i^{t-\Delta t}(\phi,\lambda +
   \Delta \lambda)+ \Psi_i^{t-\Delta t}(\phi,\lambda - \Delta \lambda)+
   6\Psi_i^{t-\Delta t}(\phi,\lambda)]\}
\end{align}
%
and afterwards a zonal smoothing is applied
%
\begin{align}
  \Psi_i^{t-\Delta t, smo}& = \Psi_i^{t-\Delta t, \lambda_s} - 0.03
  \{\Psi_i^{t-\Delta t, \lambda_s}(\phi+ 2\Delta \phi,\lambda +
  ) + \Psi_i^{t-\Delta t, \lambda_s}(\phi- 2\Delta \phi,\lambda ) - \notag \\
  & 4 [\Psi_i^{t-\Delta t, \lambda_s}(\phi+ \Delta \phi,\lambda )+
  \Psi_i^{t-\Delta t, \lambda_s}(\phi-\Delta \phi,\lambda )+
   6\Psi_i^{t-\Delta t, \lambda_s}(\phi,\lambda)]\}
\end{align}
%
for $i = 1, 2$ which corresponds to $O_2$ and $O$, respectively. The
horizontal diffusion factor is set by
%
\begin{align}
  |\lambda| & > 40^o    & \qquad d_{fac} = & 0.25 + 1.0 \notag \\
  |\lambda| & \leq 40^o & \qquad d_{fac} = & 0.25 + \frac{1}{2}
        \left( 1+\sin(\frac{\Pi (|\lambda| - 20^o)}{40^o})\right)
\end{align}
%
If the input flag $DIFHOR$ is set to zero, in the input file then
$d_{fac} = 0$. The mean molecular mass $\overline{m}$ is calculated
at the midpoint level
%
\begin{align}
 \overline{m} (z+\frac{1}{2}\Delta z) = \left( \frac{\Psi(O_2)}{m_{O_2}} +
    \frac{\Psi(O)}{m_{O}}+ \frac{\Psi(N_2)}{m_{N_2}} \right)^{-1}
\end{align}
%
with $\Psi(N_2) = 1 - \Psi(O_2) - \Psi(O)$. Note that the mass
mixing ratios are also on the midpoint levels. \\

The lower boundary values for the mass mixing ratios are set by
%
\begin{align}
  \Psi_i(z_{LB}) = b(i,1) \Psi(O_2)(z_{LB}+\frac{1}{2}\Delta z) +
   b(i,2) \Psi(O)(z_{LB}+\frac{1}{2}\Delta z) + f_b(i)
\end{align}
%
for $i = 1, 2$ which corresponds to $O_2$ and $O$, respectively. The
values $b(i,1)$, $b(i,2)$ and $f_b(i)$ are set in the
\src{boundary\_module}. With the mass mixing ratios at the lower
boundary the mean molecular mass can be determined at the lower
boundary
%
\begin{align}
 \overline{m} (z_{LB}) = \left( \sum_{i=1}^3 \frac{\Psi_i}{m_i}(z_{LB})  \right)^{-1}
\end{align}
%
for $i = 1, 2 \text{and} 3$ which corresponds to $O_2$, $O$ and
$N_2$, respectively. The term $\frac{1}{\overline{m}}\frac{\partial
\overline{m}}{\partial z}$ at the bottom is determined
%
\begin{align}
   wk_4  = \frac{1}{\overline{m}(z_{LB}) +\overline{m}(z_{LB}+\frac{1}{2}\Delta z) }
      \frac{\overline{m}(z_{LB}+\frac{1}{2}\Delta z) +\overline{m}(z_{LB})}{\Delta z}
\end{align}
%
???? where is the factor of 1/2 and 1/2 in $wk_4$ which level. Is
the LB below the LB level???????. The $epsilon $terms $1-
\frac{1}{\overline{m}}[{m_i} + \frac{\partial \overline{m}}{\partial
z}]$ are
%
\begin{align}
   \epsilon_{i}(k_m) = 1- \frac{2}{\overline{m}(z_{LB}) +
       \overline{m}(z_{LB}+\frac{1}{2}\Delta z)}[{m_i} +
     \frac{\overline{m}(z_{LB}+\frac{1}{2}\Delta z) -
      \overline{m}(z_{LB})}{\Delta z}]
\end{align}
%
for $i = 1, 2$ which corresponds to $O_2$ and $O$, respectively, and
$k_m = 2$. The value
%
\begin{align}
  zz_{1}(k_m) = zz_{1}(k_m) = 0
\end{align}
%
??? how does the algorithm goes????. Next the matrix
$\mathbf{\alpha}$ at the height level $z+0.25\Delta z$ is determined
%
\begin{align}
   \alpha^*_{1,1}(k_m) = & \alpha_{22} = -\Delta_{11}[\phi_{2,3}+\{\phi_{2,1}-\phi_{2,3}\} \notag \\
                         & \frac{1}{2} \{ \Psi_1(z_{LB}) + \Psi_1(z_{LB}+\frac{1}{2}\Delta
                         z)\}] \\
   \alpha^*_{1,2}(k_m) = & -\alpha_{12} = [1-\Delta_{12}][\phi_{1,2}- \phi_{1,3}] \notag \\
                         & \frac{1}{2} \{ \Psi_1(z_{LB}) + \Psi_1(z_{LB}+\frac{1}{2}\Delta
                         z)\}] \\
   \alpha^*_{2,2}(k_m) = & \alpha_{11} = -\Delta_{22}[\phi_{1,3}+\{\phi_{1,2}-\phi_{1,3}\} \notag \\
                         & \frac{1}{2} \{ \Psi_2(z_{LB}) + \Psi_2(z_{LB}+\frac{1}{2}\Delta
                         z)\}]\\
   \alpha^*_{2,1}(k_m) = & -\alpha_{21} = [1-\Delta_{22}][\phi_{2,2}- \phi_{2,3}] \notag \\
                         & \frac{1}{2} \{ \Psi_2(z_{LB}) + \Psi_2(z_{LB}+\frac{1}{2}\Delta z)\}]
\end{align}
%
Note, that the indices in the source code already take into account
the inverse of the matrix with $\mathbf{\alpha}^{-1} =
\mathbf{\alpha}^*/det|\mathbf{\alpha}| $, and therefore the diagonal
values are switches and the off-diagonal terms have an minus sign.
The determinant $det|\mathbf{\alpha}|$ is the same as
$det|\mathbf{\alpha}^*(k_m)|=
\alpha^*_{1,1}(k_m)\alpha^*_{2,2}(k_m)-\alpha^*_{1,2}(k_m)\alpha^*_{2,1}(k_m)$.
The matrix $\mathbf{\alpha}^{*}$ is multiplied by the factor
$\frac{1}{\tau} \frac{\overline{m}}{m_{N_2}}\left(
\frac{T_{00}}{T_n}\right)^{0.25}\frac{1}{det|\mathbf{\alpha}^*|}$
%
\begin{align}
   wk_1 = \frac{1}{\tau}\frac{1}{2} [\overline{m}(z_{LB}+ \overline{m}(z_{LB}+\frac{1}{2}\Delta z)]
   \frac{1}{m_{N_2}}\left(
\frac{T_{00}}{T_n(z_{LB})}\right)^{0.25}\frac{1}{det|\mathbf{\alpha}^*(k_m)|}
\end{align}
%
 which lead to the quantity defined as $A = \frac{1}{\tau} \frac{\overline{m}}{m_{N_2}}\left(
\frac{T_{00}}{T_n}\right)^{0.25}\mathbf{\alpha}^{-1}$. The value
$\gamma$ at the bottom are set to
%
\begin{align}
  \gamma_{ij}(k=1) = 0.
\end{align}
%
for $i = 1, 2$ and $j+ 1, 2$, which corresponds to $O_2$ and $O$,
respectively. \\

After setting all the values at the bottom of the model, the height
loops starts from the bottom to the top of the model ($k=1/ z=
-6.75$ to $k = nlev-1 / z= 6.75$). The $epsilon $terms $1-
\frac{1}{\overline{m}}[{m_i} + \frac{\partial \overline{m}}{\partial
z}]$ are set at $z$ by
%
\begin{align}
   \epsilon_{i}(k_p) = 1- \frac{2}{\overline{m}(z- \frac{1}{2} \Delta z) +
       \overline{m}(z+\frac{1}{2}\Delta z)}[{m_i} +
     \frac{\overline{m}(z+\frac{1}{2}\Delta z) -
      \overline{m}(z-\frac{1}{2}\Delta z)}{\Delta z}]
\end{align}
%
for $i = 1, 2$ which corresponds to $O_2$ and $O$, respectively, and
$k_p = 1$. Note that before the value $\epsilon_{i}(k_m)$ was
determined, with $k_m = 2$. The values of $k_m$ and $k_p$ are
alternating between $1$ and $2$ to get the values at the previous
height level $k_m= k-1$, and the height level $k_p = k+1$. The
$\mathbf{\alpha}$ is determined at $k_p / z$ by
%
\begin{align}
   \alpha^*_{1,1}(k_p) = & \alpha_{22} = -\Delta_{11}[\phi_{2,3}+\{\phi_{2,1}-\phi_{2,3}\} \notag \\
                         & \frac{1}{2} \{ \Psi_1(z-\frac{1}{2}\Delta z) + \Psi_1(z+\frac{1}{2}\Delta
                         z)\}] \\
   \alpha^*_{1,2}(k_p) = & -\alpha_{12} = [1-\Delta_{12}][\phi_{1,2}- \phi_{1,3}] \notag \\
                         & \frac{1}{2} \{ \Psi_1(z-\frac{1}{2}\Delta z) + \Psi_1(z+\frac{1}{2}\Delta
                         z)\}] \\
   \alpha^*_{2,2}(k_p) = & \alpha_{11} = -\Delta_{22}[\phi_{1,3}+\{\phi_{1,2}-\phi_{1,3}\} \notag \\
                         & \frac{1}{2} \{ \Psi_2(z-\frac{1}{2}\Delta z) + \Psi_2(z+\frac{1}{2}\Delta
                         z)\}] \\
   \alpha^*_{2,1}(k_p) = & -\alpha_{21} = [1-\Delta_{22}][\phi_{2,2}- \phi_{2,3}] \notag \\
                         & \frac{1}{2} \{ \Psi_2(z-\frac{1}{2}\Delta z) + \Psi_2(z+\frac{1}{2}\Delta z)\}]
\end{align}
%
The matrix $\mathbf{\alpha}^{*}$ is multiplied by the factor
$\frac{1}{\tau} \frac{\overline{m}}{m_{N_2}}\left(
\frac{T_{00}}{T_n}\right)^{0.25}\frac{1}{det|\mathbf{\alpha}^*|}$
%
\begin{align}
   wk_1 = \frac{1}{\tau}\frac{1}{2} [\overline{m}(z- \frac{1}{2}\Delta )\overline{m}(z+\frac{1}{2}\Delta z)]
   \frac{1}{m_{N_2}}\left(
\frac{T_{00}}{T_n(z)}\right)^{0.25}\frac{1}{det|\mathbf{\alpha}^*(k_p)|}
\end{align}
%
 which lead to the quantity defined as $A = \frac{1}{\tau} \frac{\overline{m}}{m_{N_2}}\left(
\frac{T_{00}}{T_n}\right)^{0.25}\mathbf{\alpha}^{-1}$. The term
$\frac{1}{\overline{m}}\frac{\partial \overline{m}}{\partial z}$ at
is determined, but first the value  $wk_4$ which holds the value for
$k_p / z$ is copied into $wk_3$ which holds the value for $k_m /
z-\Delta z$.
%
\begin{align}
   wk_4  = \frac{1}{\overline{m}(z-\frac{1}{2}\Delta z) +\overline{m}(z+\frac{1}{2}\Delta z) }
      \frac{\overline{m}(z+\frac{1}{2}\Delta z) +\overline{m}(z-\frac{1}{2}\Delta z)}{\Delta z}
\end{align}
%
???? where is the factor of 1/2 and 1/2 in $wk_4$ which level.
???????. \\

The discretizied equation can be written as
%
\begin{align}
  \mathbf{P}_k \mathbf{\Psi}_{k-1}^{t_n+ \Delta t} +
  \mathbf{Q}_k \mathbf{\Psi}_{k}^{t_n+ \Delta t} +
  \mathbf{R}_k \mathbf{\Psi}_{k+1}^{t_n+ \Delta t} = \mathbf{f}_k
\end{align}
%
with
%
\begin{align}
  P_k = & \frac{A(k_m)}{\Delta z} \left(\frac{1}{\Delta z} + \frac{\epsilon(k_m)}{2} \right)
  - \frac{e^{-z} K(k_m)}{\Delta^2 z} - \frac{e^{-z} W(k)}{2 \Delta
  z} \\
  R_k = & \frac{A(k_p)}{\Delta z} \left(\frac{1}{\Delta z} + \frac{\epsilon(k_p)}{2} \right)
  - \frac{e^{-z} K(k_p)}{\Delta^2 z} + \frac{e^{-z} W(k)}{2 \Delta
  z} \\
  Q_k =&  - [\frac{A(k_m)}{\Delta z} \left(\frac{1}{\Delta z} - \frac{\epsilon(k_m)}{2} \right)  \\
    & \frac{A(k_p)}{\Delta z} \left(\frac{1}{\Delta z} + \frac{\epsilon(k_p)}{2} \right)
  - \frac{e^{-z} K(k_m)}{\Delta^2 z}- \frac{e^{-z} K(k_p)}{\Delta^2 z} + JSe^{-z} -
   \frac{e^{-z}}{2 \Delta t} ] \notag \\
   f_k = & e^{-z}[ \frac{\Psi^{t_n-\Delta t}(k)}{2 \Delta t} - \mathbf{v}_n \cdot \nabla \Psi^{t_n}(k)]
\end{align}
%
In the source code the values are
%
\begin{align}
  P_k(i,j) = & [{A(k_m)(i,j)}
   \left(\frac{1}{\Delta z} + \frac{\epsilon(k_m)(j)}{2} \right)
      \notag \\
  & - e^{-(z+\frac{1}{2}\Delta z)}( e^{\frac{1}{2}\Delta z}K(k???z-\frac{1}{2}\Delta z)d_{fac}
    (\frac{1}{\Delta z}-wk_3) + \notag \\
  & 0.25 (W(z)+W(z+\Delta z))) \Delta(i,j)] \frac{1}{\Delta z}
\end{align}
%
\begin{align}
  R_k(i,j) = & [A(k_p)(i,j)
   \left(\frac{1}{\Delta z} - \frac{\epsilon(k_p)(j)}{2} \right)
      \notag \\
  & - e^{-(z+\frac{1}{2}\Delta z)}( e^{-\frac{1}{2}\Delta z}K(k+1/z+\frac{1}{2}\Delta z)d_{fac}
  (\frac{1}{\Delta z}+wk_4) -  \notag \\
  &  0.25 (W(z)+W(z+\Delta z))) \Delta(i,j)] \frac{1}{\Delta z}
\end{align}
%
\begin{align}
  Q_k(i,j) = & - \frac{1}{\Delta z}[A(k_m)(i,j)
   \left(\frac{1}{\Delta z} - \frac{\epsilon(k_m)(j)}{2} \right) +
   A(k_p)(i,j)
   \left(\frac{1}{\Delta z} + \frac{\epsilon(k_p)(j)}{2} \right)
      \notag \\
  & + e^{-(z+\frac{1}{2}\Delta z)}( \{e^{\frac{1}{2}\Delta z}K(k+1/z+\frac{1}{2}\Delta z)
  (\frac{1}{\Delta z}-wk_4) -  \notag \\
  & e^{-\frac{1}{2}\Delta z}K(k/z-\frac{1}{2}\Delta z)
  (\frac{1}{\Delta z}+wk_3) \}\frac{d_{fac}}{\Delta z} + \frac{1}{2\Delta t}\notag \\
  &) \Delta(i,j) - fs(i,j)]
\end{align}
%
The right hand side is assembled in the following
%
\begin{align}
  f_k(i) =e^{-(z+\frac{1}{2}\Delta z)}[ \frac{\Psi_i^{t_n-\Delta t,smo}}{2 \Delta t}
     -\mathbf{v}_n\cdot \nabla \Psi_i + f_{s,0}(i) + h_{d,i}]
\end{align}
%
for $i = 1,2$ corresponding to $O_2$ and $O$, respectively. \\

At the lower boundary the coefficients are adjusted ??? what's the
condition there???
%
%
\begin{align}
 Q_k^*(i,j)(z_{bot}) = Q_k(i,j)(z_{bot}) +P_k(i,1)(z_{bot})b(1,j) +
      P_k(i,2)(z_{bot})b(2,j)
\end{align}
%
with $z_{bot} = -6.75$ or $z_{bot} = z_{LB}+ \frac{1}{2}\Delta z$,
and $i = 1, 2$ and $j=1, 2$ corresponding to $O_2$ and $O$,
respectively. The updated right hand side is
%
\begin{align}
 f_k^*(i)(z_{bot}) = f_k(i)(z_{bot}) -P_k(i,1)(z_{bot})fb(1) -
      P_k(i,2)(z_{bot})fb(2)
\end{align}
%
with $i = 1, 2$ corresponding to $O_2$ and $O$, respectively. \\

At the upper boundary we set $\frac{\partial}{\partial z}
\mathbf{\Psi} = \epsilon \mathbf{\Psi}$ which leads to
%
\begin{align}
 \frac{\Psi(k+1) - \Psi(k)}{\Delta z} = \frac{\epsilon(z+\frac{1}{2}\Delta z)}{2}
 (\Psi(k+1)+\Psi(k)))
\end{align}
%
with $k= nlev1-1$ and $k+1=nlev$ corresponding to
$z_{ub}-\frac{3}{2}\Delta z=6.25$ and $z_{ub}-\frac{1}{2}\Delta
z=6.75$, respectively. The coefficient equation looks like
%
\begin{align}
  P_k \Psi(k-1) + [Q_k + R_k \frac{1 + 0.5 \epsilon(k+1/2) \Delta z}
  {1 - 0.5 \epsilon(k+1/2) \Delta z}]\Psi(k) = f_k
\end{align}
%
??? make matrix out of the equation????
%
%
\begin{align}
 Q_k^*(i,j)(z_{ub}-\frac{3}{2}\Delta z) & = Q_k(i,j)(z_{ub}-\frac{3}{2}\Delta z)
 + \notag \\
   & \frac{1 + 0.5 \epsilon(k+1/2) \Delta z}{1 - 0.5 \epsilon(k+1/2) \Delta
   z} R_k(i,j)(z_{ub}-\frac{3}{2}\Delta z) \\
  R_k^*(i,j)(z_{ub}-\frac{3}{2}\Delta z) & = 0
\end{align}
%
with $i = 1, 2$ and $j=1, 2$ corresponding to $O_2$ and $O$,
respectively.
%
For solving the equation the following values are needed ??? is this
Newton iteration??? How does it work????
%
\begin{align}
  Q_k'(i,j) = & Q_k(i,j) - P_k(i,1)*\gamma(1,j)(k)-
     P_k(i,2)*\gamma(2,j)(k) \\
  wk_1    = & det|Q_k'| \\
  wk_{m1}(1,j) = & \frac{1}{wk_1} [\Delta(1,j)Q_k(2,2) - (1-\Delta(1,j))Q_k(1,j)] \\
  wk_{m1}(2,j) = & \frac{1}{wk_1} [\Delta(2,j)Q_k(1,1) - (1-\Delta(2,j))Q_k(2,j)] \\
  wk_{v1}(i) = & f_k(i) - \sum_{j=1}^2 P_k(i,j)zz(k)(j) \\
  \gamma(k+1)(1,j) = & \sum_{m=1}^2 wk_{m1}(1,m) R_k(m,j)\\
  \gamma(k+1)(2,j) = & \sum_{m=1}^2 wk_{m1}(2,m) R_k(m,j) \\
  zz(k+1)(i) = &  \sum_{m=1}^2 wk_{m1}(i,m) wk_{v1}(m)
\end{align}
%
Hereby end the height loop from the bottom to the top of the model.
The values at the top ($k=nlev+1$ or $z_{UB}+\frac{1}{2}\Delta z =
7.25$) are set
%
\begin{align}
   \Psi_i^{t_n+\Delta t}(z_{UB}+\frac{1}{2}\Delta z) = 0
\end{align}
%
with $i = 1, 2$ corresponding to $O_2$ and $O$, respectively. The
updated value is
%
%
\begin{align}
  upd_i(z_{UB}+\frac{1}{2}\Delta z)  = 0
\end{align}
%
Then the downward sweep begins
%
\begin{align}
   upd_i(k)  = & zz(k+1)(i) \\
   upd_i(k)  = & upd_i(k) - \sum_{j=1}^2 \gamma(k+1)(i,j)upd_j(k+1) \\
   \Psi_i^{t_n+\Delta t}(k) = & upd_i(k)
\end{align}
%
with $k$ and $k+1$ corresponding to $z-\frac{1}{2}\Delta z$ and
$z+\frac{1}{2}\Delta z$, respectively, and with $i = 1, 2$ and $j=1,
2$ corresponding to $O_2$ and $O$, respectively. \\

The upper boundary is set by
%
\begin{align}
  \Psi_i^{t_n+\Delta t}(lev1) = \frac{1 + 0.5 \epsilon_i(k_p) \Delta z}
      {1 - 0.5 \epsilon_i(k_p) \Delta z}\Psi_i^{t_n+\Delta t}(lev1-1)
\end{align}
%
with $i = 1, 2$ and corresponding to $O_2$ and $O$, respectively. \\
%
The calculated values for the mass mixing ratio $\Psi_i^{upd, t +
\Delta t}$ are smoothed by a Fast Fourier transformation. All the
wave numbers larger than a predefined value at each latitude are
removed. The wave numbers are defined in the module \src{cons.F}.
The values of the mass mixing ratio at the timestep $t_n$ are also
updated by using
%
\begin{align}
  \Psi_i^{upd,t_n} = \frac{1}{2}({1-c_{smo}})(\Psi_i^{t_n-\Delta t}+
     \Psi_{i,{smo}}^{upd,t_n+\Delta t}) + c_{smo}\Psi_i^{t_n}
\end{align}
%
with $c_{smo} = 0.95$, and $i = 1, 2$ and corresponding to $O_2$ and
$O$, respectively. Non negative values are assured by
%
\begin{align}
  \Psi_i^{upd,t_n} < & 1 \times 10^{-6} &  \quad & \Psi_i^{upd,t_n} = 1 \times
  10^{-6} \\
  \Psi_i^{upd,t_n+\Delta t} < & 1 \times 10^{-6} & \quad &\Psi_i^{upd,t_n+\Delta t} = 1 \times
  10^{-6} \\
  1-\sum_{i=1}^2 \Psi_i^{upd,t_n+\Delta t} < & 1 \times 10^{-6} &  \quad &
          \Psi_i^{upd,t_n+\Delta t} =
          \Psi_i^{upd,t_n+\Delta t}\frac{1-10^{-6}}{\sum_{i=1}^2 \Psi_i^{upd,t_n+\Delta t}} \\
  1-\sum_{i=1}^2 \Psi_i^{upd,t_n} < & 1 \times 10^{-6} &  \quad &
          \Psi_i^{upd,t_n} =
          \Psi_i^{upd,t_n}\frac{1-10^{-6}}{\sum_{i=1}^2 \Psi_i^{upd,t_n}}
\end{align}
%
