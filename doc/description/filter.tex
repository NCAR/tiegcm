%
\section{\index{Shapiro Filter}}\label{cap:shapiro}
%
A Shapiro Filter/smoother is used for  
time-dependent dynamical variables $T_n, U_n, V_n, O^+, \Psi_O, \Psi_{O_2},
\Psi_{NO}$  and  $\Psi_{N(^4S)}$
in the TIEGCM to limit the development of numerical 
nonlinear instability and short wavelength waves 
(less than four grid size) that are poorly represented 
in the model. Shapiro smoothing is used in 
modules/subroutines that solve equations for 
these variables. \\
%
\textbf{*Note: the following material is from \cite{wang1998}, 
and copyright to the University of Michigan. Some of 
the material has been modified based on the updated TIEGCM. } \\
%
In real world, wave interactions occur that generate 
larger and smaller wavelength waves. The smaller waves 
cascade in sizes to reach the characteristic scale of 
molecular dissipation which is the process that finally 
eliminates motion. However, in a numerical model that has 
discrete grid, motion energy cascade from larger wavelength 
waves to smaller and smaller scales is interrupted. In fact, 
waves with wavelength smaller than $2 \Delta x$ ($\Delta x$ is the grid size 
of a numerical model) are erroneously represented as larger 
wavelength waves. This phenomenon is called aliasing 
\cite{pielke1984}. The net result of this aliasing is a 
fictitious energy buildup since energy is added 
continuously to the model through forcing terms, 
while energy dissipation processes are cut off. 
Therefore, even though a numerical scheme is 
linearly stable, the results can degrade into physically 
meaningless computational noise. The solutions of model 
dynamical variables can grow to infinity. This numerical 
error is commonly referred to as nonlinear instability. \\
%
To remove aliasing in the wave presentation, and thus to 
prevent the nonlinear instability from occurring, short 
waves with wavelength smaller than $4 \Delta x$ must be eliminated 
from a numerical model. In addition to the above mentioned 
nonlinear instability problem, such short waves are also 
poorly represented in terms of phase and amplitude. And they 
are expected to dissipate to even smaller scale motions anyway. 
Therefore, it is desirable to remove these waves completely.\\
%
In the TIEGCM the smoothing technique discussed by \cite{shapiro1970} 
is used to control computational noise. The Shapiro smoother is 
a low-pass filter that eliminates waves with wavelength smaller 
than $4 \Delta x$ each time step, but leaves the large-scale disturbances 
unaffected. Another way to control nonlinear instability is to 
enhance the horizontal eddy diffusion by proper parameterization 
of the subgrid correlation terms. This method is not desirable 
since it allows modelers to adjust the eddy diffusion coefficient 
arbitrarily to change the magnitude of numerical solutions 
\cite{tag1979},\\
%
 A one-dimensional, five-point Shapiro smoothing scheme is 
 used in both the TIEGCM to eliminate or constrain any spurious, 
 nonlinear growth of high-frequency waves that may be introduced 
 by roundoff and truncations errors or by the interruption of the 
 energy cascade process that transport energy from large-scale 
 disturbances to small-scale motions. Any time-dependent 
 dynamical variable Z( $T_n, U_n, V_n, O^+, \Psi_O, \Psi_{O_2},
\Psi_{NO}$  and  $\Psi_{N(^4S)}$) is averaged each time 
 step by
%
\begin{equation}
  \bar{Z} = Z_l - \alpha(Z_{l+2}+ Z_{l-2}-4 Z_{l+1}+6 Z_{l})
    \label{eq:filer_1}
\end{equation}
% 
where $l$ is the index of grid point in either latitude or 
longitude direction.$\alpha$  is a smoothing factor and $0 < \alpha < 0.5$.
$Z(x)$  can 
be expressed in terms of a summation of Fourier components of the form
%
\begin{equation}
  Z_l = C+A cos k(x_l + \varphi)
    \label{eq:filer_2}
\end{equation}
% 
where $C$ is a constant, $A$ is the amplitude of the wave 
component with wave number $k$ ( $k=2\Pi/\lambda$, where 
$\lambda$  is the wavelength of 
the component). Substituting \ref{eq:filer_2} into \ref{eq:filer_1} yields
%
\begin{equation}
  \begin{split}
  \bar{Z} = & C + A cos k(x_l + \varphi) [1- 2 \alpha(cos 2 k \Delta x - 4 cos k
  \Delta x + 3)]  \\
  =&  C + A[1-4 \alpha(1-cos k \Delta x)^2]cos k(x_l + \varphi)
  \end{split}
    \label{eq:filer_3}
\end{equation}
% 
$\alpha$ is set to be equal to 0.03. $\alpha$  values are determined 
 by numerical experiments. It is desirable to have $\alpha$  as 
 small as possible so that longer waves are less affected 
 by the smoothing. The minimum value of $\alpha$  is obtained by 
 gradually decreasing $\alpha$  while preserving integrity and 
 stability of the numerical solutions. 
% 
It is clear from \ref{eq:filer_3} that all wavelengths are 
damped. However, short waves ($\lambda , 8 \Delta x$) are heavily damped while long 
waves are almost unaffected. Successive application of this 
smoother on a dynamical variable prevents short waves from 
growing to such a degree that they degrade the numerical 
solutions.
%
\section{\index{Fourier Filter}}\label{cap:fourier}
%
\textbf{*Note: the following material is from \cite{wang1998}, 
and copyright to the University of Michigan. Some of 
the material has been modified based on the updated TIEGCM.} \\
%
Since the TIEGCM uses a uniform horizontal grid system 
and finite-difference numerical scheme, computational 
errors can grow significantly as the grids approach to 
the poles. The zonal grid sizes decrease in direct 
proportion to $cos \phi$ , where $\phi$ is the latitude. The truncation 
error of the finite-difference scheme and other numerical 
errors may be amplified by the zonal finite difference 
calculations at the grid points near poles. Non-physical 
waves are then generated and grow non-linearly to cause 
computational instability. This pole problem occurs in any 
simulation that uses finite difference techniques with uniform 
latitude and longitude grids. One way to avoid this problem is 
to use much smaller time steps, which is undesirable because it 
increases the computational cost dramatically by slowing down the 
entire simulation process. \\
%
Another way to solve this pole problem is to apply a Fourier 
filter at high latitudes to remove non-physical high-frequency 
zonal waves generated by finite differences in the region with 
smaller zonal grid sizes. At each time step a Fourier expansion 
is applied to the prognostic variables of the model. A cutoff 
frequency was found at each latitude (at high latitudes) as a 
result of numerical experiments. Waves with frequencies that are 
higher than the cutoff frequencies are eliminated from the Fourier 
spectra of prognostic variables. The values of prognostic variables 
are then recovered through a reverse Fourier transform using modified 
Fourier spectra. Each latitude grid may have its own cutoff frequency. 
The cutoff frequency at a particular latitude is determined by the 
expected time step used by a model. A large time step requires low 
cutoff frequencies and more zonal waves are filtered out.\\
% 
In the TIEGCM, Fourier filters are applied to the individual 
prognostic variables. These variables include  $T_n, U_n, V_n, O^+, 
\Psi_O, \Psi_{O_2},
\Psi_{NO}, \Psi_{N(^4S)}$  and  $W$. 
It should be noted here that to apply a Fourier filter directly and 
successively on prognostic variables may cause overfiltering which 
may remove fine structures that are expected to occur at high latitudes.

 


