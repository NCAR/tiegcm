%
\section{Calculation vertical velocity $W$ / \index{SWDOT.F},  \index{DIVRG.F}}\label{cap:divrg}
%
The input to \src{subroutine swdot} is summarized in table
\ref{tab:input_swdot}.
%
\begin{table}[tb]
\begin{tabular}{|p{3.5cm} ||c|c|c|c|c|c|} \hline
physical field               & variable        & unit&pressure
level& timestep
\\ \hline \hline
%
neutral zonal velocity &       $u_n$              & $cm/s$   &  midpoints & $t_n$\\
neutral meridional velocity &       $\cos \lambda v_n$              & $cm/s$   &  midpoints & $t_n$\\
 \\ \hline
\end{tabular}
\caption{Input fields to \src{subroutine swdot}}
\label{tab:input_swdot}
\end{table}
%
The output of \src{subroutine swdot} is summarized in table
\ref{tab:output_swdot}.
%
\begin{table}[tb]
\begin{tabular}{|p{3.5cm} ||c|c|c|c|c|c|} \hline
physical field               & variable        & unit&pressure
level& timestep \\ \hline \hline 'dimensionless' vertical velocity &
$W$ & $1/s$ & interfaces  & $t_n+\Delta t$
\\ \hline \hline
\end{tabular}
\caption{Output fields of \src{subroutine swdot}}
\label{tab:output_swdot}
\end{table}
%
The vertical velocity is calculated by solving the continuity
equation assuming linearity and incompressibility of the
thermospheric neutral gas. The continuity equation takes the
following form
%
\begin{align}
  \frac{1}{R \cos \lambda} \frac{\partial}{\partial \lambda} (\cos \lambda
  v_n) + \frac{1}{R \cos \lambda } \frac{\partial u_n}{\partial
  \phi} + e^z \frac{\partial}{\partial Z}(e^{-z}W) = 0
\end{align}
%
with the 'dimensionless' vertical velocity is $W= \frac{dZ}{dt}$.
The 'real' vertical velocity $w$ is obtained by integrating the
continuity equation over $Z$ to get $W$, and then multiply $W$ by
the scale height $H$. \\

The horizontal divergence $\nabla_H \cdot \mathbf{v}_n $ is
calculated in \src{subroutine divrg}.
%
\begin{align}
  \nabla_H \cdot \mathbf{v}_n = & \frac{1}{R_E \cos \lambda} [ \frac{2}{3 \Delta \phi}\left[ u_n(\phi+ \Delta \phi,\lambda)
    u_n(\phi- \Delta \phi,\lambda)\right] - \notag \\
    {}&  \frac{1}{12 \Delta \phi}\left[ u_n(\phi+ 2\Delta \phi,\lambda)
    u_n(\phi- 2\Delta \phi,\lambda)\right] \notag \\
  {} & \frac{2}{3 \Delta \phi}\left[ \cos (\lambda + \Delta \lambda) v_n(\phi,\lambda + \Delta \lambda)
    cos (\lambda - \Delta \lambda) v_n(\phi,\lambda - \Delta \lambda)\right]
    - \notag \\
   {} & \frac{1}{12 \Delta \lambda}\left[ v_n(\phi,\lambda+2\Delta \lambda)
    v_n(\phi,\lambda-2 \Delta \lambda)\right]]
\end{align}
%
The integration is done by an integration from the top to the bottom of
the model, with the condition at the upper boundary being
%
\begin{align}
  \frac{\partial w}{\partial z} = 0
\end{align}
%
and in descritized form
%
\begin{align}
  W(z_{top}) = \nabla_H \cdot \mathbf{v}_n(z_{top}-\frac{1}{2} \Delta z)
\end{align}
%
and then do the integration ?????
%
\begin{align}
  W(z - \Delta z) = e^{-\frac{1}{2}z}\left[e^{-\frac{1}{2}z}W(z - 2\Delta z)+ \Delta z
  \nabla_H \cdot \mathbf{v}_n(z+\frac{1}{2} \Delta
  z)\right]
\end{align}
%
The vertical velocity $W$ is then filtered in longitude to remove
the high wave numbers. The filter is in \src{subroutine filter\_w}
and is part of the file \src{swdot.F}.
