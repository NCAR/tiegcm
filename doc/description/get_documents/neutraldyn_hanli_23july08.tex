The TIE-GCM solves the primitive equations in pressure coordinate, and the detailed information on
the equations and the numerics employed can be found in \cite{dickinsonetal81}. In this Chapter,
we will first give a brief summary of the model dynamics, and then discuss the main components
in solving dynamic equations. The solution of the thermodynamics will be discussed in the next
Chpater.

The zonal and meridional winds are solved from the momentum equations in the zonal and meridional
direction, respectively. The tendencies (acceleration) of the horizontal winds come from horizontal
and vertical momentum transport, horizontal gradient of geopotential, eddy and molecular viscosity,
Coriolis force, and ion drag (Section \ref{cap:duv}). The vertical wind is determined from the
continuity equation by integrating the divergence of the horizontal winds vertically (Section
\ref{cap:divrg}). The geopotential is calculated from the hydrostatic equation by integrating
temperature vertically (Section \ref{cap:addiag}).

The equations are solved using 4th order finite difference method. The vertical grid is staggered
with horizontal winds, neutral, ion, and electron temperature, and mass mixing ratios defined
on the so-called midpoints, and vertical wind and geopotential defined on the interfaces.
Leapfrog scheme is used for time integration of momentum transport, geopontential gradient,
Coriolis force, and ion drag, and implicit scheme is used for eddy and molecular viscosity
in the vertical direction. To achieve better numerical stability, Shapiro filter (reference?) is
applied in meridional and zonal directions.
