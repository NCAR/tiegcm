\chapter{Introduction}\label{cap:intro}
%
The High Altitude Observatory at the National Center for Atmospheric Research 
has developed a series of numeric simulation models of the Earth's upper 
atmosphere, including the upper Stratosphere, Mesosphere, and Thermosphere. 
The Thermospheric General Circulation Models (TGCM's) are three-dimensional, 
time-dependent models of the Earth's neutral upper atmosphere. The models use a 
finite differencing technique to obtain a self-consistent solution for the coupled, 
nonlinear equations of hydrodynamics, thermodynamics, continuity of the neutral gas 
and for the coupling between the dynamics and the composition. \\

Models in the series include a self-consistent aeronomic scheme for the 
coupled Thermosphere/Ionosphere system, the Thermosphere Ionosphere Electrodynamic 
General Circulation Model (TIEGCM), and an extension of the lower boundary from 97 
to 30 km, including the physical and chemical processes appropriate for the Mesosphere 
and upper Stratosphere, the Thermosphere Ionosphere Mesosphere Electrodynamic General 
Circulation Model (TIME-GCM). A global mean, or column model, has also been developed 
in parallel with the TGCM's. The global mean model is used as a time-dependent, 
one-dimensional platform from which new chemical, dynamic and numeric schemes are 
developed and tested before being introduced into the 3-d GCM's.  \\


This documentation presents the governing equations, physical 
parameterizations and numerical algorithms used in the NCAR 
Thermosphere Ionosphere Electrodynamics GCM (TIE-GCM) version 1.94. The model description
provides help on the major model component, numerics and coding implementation.
 \\

\vspace{1cm}
%
\textbf{Conventions} \\
%
\begin{tabular}{l  l } 
\command{command}	        &   bold \\
\replaceable{template filenames}&   italics \\       
\flags{flags}	                &   medium bold \\
\directory{directories, files}  &   slanted \\
\src{source code}               &   typewrite  \\ 	
\myemph{keywords} 	        &   emphasize
\end{tabular}
%
%\subsection{Brief History}\label{subcap:history}
