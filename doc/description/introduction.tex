\section{Introduction}\label{cap:intro}
%
This report presents the governing equations, physical 
parameterizations and numerical algorithms used in the NCAR 
Thermosphere-Ionosphere-Electrodynamics GCM version 1.8. The model description
provides help on the major model component, numerics and coding implementation.

The electrodynamic part of TIEGCM is a serial code, which uses finite differencing to
discretize the stationary electrostatic equation. We assume that the field--lines are
equipotential which reduces the 3D equation to 2D. In addition, at 
low-- and mid--latitude hemispheric
symmetry of the electric potential is assumed and therefore the 
electrodynamo equation is only solved in one
hemisphere. At high latitudes the electric
potential is prescribed from empirical models e.g. Heelis or Weimer [literature].
 \\

\vspace{1cm}
%
\textbf{Conventions} \\
%
\begin{tabular}{l  l } 
\command{command}	        &   bold \\
\replaceable{template filenames}&   italics \\       
\flags{flags}	                &   medium bold \\
\directory{directories, files}  &   slanted \\
\src{source code}               &   typewrite  \\ 	
\myemph{keywords} 	        &   emphasize
\end{tabular}
%
%\subsection{Brief History}\label{subcap:history}
