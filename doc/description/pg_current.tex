\section{Gravity and plasma pressure driven current}
%
\subsection{Gravity Driven Current}\label{subsec:grav_current}
%
This section refers to the \texttt{subroutine magpres\_grav} in TIEGCM. Note that
the units in this section are the units used in the source code of
TIEGCM. By default the gravity driven current is calculated for a model run.
To omit the gravity and plasma pressure driven current the flag \flags{j\_pg=.false.} in \src{magpres\_g\_module} 
has to be set. \\

%
The current driven by gravity can be calculated by
%
\begin{equation}
  \mathbf{J}_g (h) =  \frac{1}{\mathbf{B}(h)^2}\rho_{ion} \mathbf{g}(h) \times \mathbf{B}(h) 
  		\label{eq:j_g}
\end{equation}
%
with $\mathbf{B}$  [Gauss] the Earth main magnetic field, the ion density 
$\rho_{ion}$ [$\frac{1}{cm^3}$], and $\mathbf{g}(h)$  [$\frac{cm}{s^2}$] 
the gravitational acceleration
at height h. The gravitational 
field gets weaker
with height and therefore the  gravitational acceleration at reference height $h_0= 90
\; km$ is scaled by 
%
\begin{equation}
  \mathbf{g}(h)  =  \biggl(\frac{R_0}{R} \biggr)^2\mathbf{g}(h_0) \label{eq:gh}
\end{equation}
%
with $R = R_E + h$ and $R_E$ the mean Earth radius. The radius of
the reference height is $R_0 = R_E + h_0$.
The ion density is determined by 
%
\begin{equation}
  \rho_{ion} = M_i \sum_i n_i m_i  \quad \text{with} \quad i= O^+,O^{2+},N^+,N^{2+},NO^+ 
  		\label{eq:iondensity}
\end{equation}
%
with the mass of a unit atomic weight $M_i$ [g], $n_i$ [$\frac{1}{cm^3}$] the ion number density 
of the species i
and $m_i$ its corresponding atomic weight. Combining equations 
(\ref{eq:gh}) and (\ref{eq:iondensity}) and converting from [$\frac{g}{s^2 cm^2}$] to 
[$\frac{kg}{s^2 m^2}$] will introduce an additional factor of 10 (see equation \ref{eq:j_gdisc}). \\
%
The height variation of the Earth main magnetic
field is approximated by
%
\begin{equation}
  \mathbf{B}(h)  =  \biggl(\frac{R_0}{R} \biggr)^3\mathbf{B}(h_0) \label{eq:bh}
\end{equation}
%
with $\mathbf{B}(h_0)$ [Gaus] referenced to $h_0$.
The components of the main field are $\mathbf{B} = (b_x,b_y,b_z)$  the
north--, east-- and downward components or $\mathbf{B} = (b_{\phi_g},b_{\lambda_g},b_{z_g})$ with
$\phi_g, \; \lambda_g \; \text{and} \; z_g$ the directions in geographic
longitude, latitude and upward height respectively. 
The current in [$\frac{A}{cm^2}$] due to the gravitational force is calculated at half pressure levels
$k+\frac{1}{2}$ by
%
\begin{equation}
  \mathbf{J}_{g,k+\frac{1}{2}} = 10 \frac{R}{R_0} \frac{1}{\mathbf{B}_0^2} \rho_{ion} 
          {g}(h_0) (b_x,-b_y,0) \label{eq:j_gdisc}
\end{equation}
%
Note that in the TIEGCM code most quantities are evaluated at half levels
$k+\frac{1}{2}$ and stored at level $k'$ with $(\cdot)'$ denoting the
half levels $(\cdot) +\frac{1}{2}$ in contrast to $k$ which is the full pressure level 
$k$. Therefore $R$ and $\rho_{ion}$ in equation (\ref{eq:j_gdisc}) must be 
evaluated at the half level $k+\frac{1}{2}$.
%
\subsection{Plasma Pressure Gradient Driven Current}\label{subsec:ppres_current}
%
This section refers to the \texttt{subroutine magpres\_grav} in TIEGCM. Note that
the units in this section are the units used in the source code of
TIEGCM. The plasma pressure gradient driven current is calculated by default for a model run.
To omit the gravity and plasma pressure driven current the flag \flags{j\_pg=.false.} in \src{magpres\_g\_module} 
has to be set. \\ \\
%
The current due to the plasma pressure gradient $\nabla p_p$ is
%
\begin{equation}
  \mathbf{J}_p  =  - \frac{1}{\mathbf{B}(h)^2}\nabla p_p \times \mathbf{B}(h) 
  		\label{eq:j_p}
\end{equation}
%
with the plasma pressure
%
\begin{equation}
  \nabla p_p  =  k_B \nabla [(T_i + T_e ) N_e] \label{eq:gradp_p}
\end{equation}
%
The Boltzmann constant is denoted by $k_B$; $T_i , \; T_e$ and $N_e$ are the ion
temperature [K], the electron temperature [K] and the electron density
[$\frac{1}{cm^3}$]. The gradient $\nabla$ is taken in the geographic direction
$(\nabla_{\phi_g}, \nabla_{\lambda_g}, \nabla_{z_g})$. \\
%
The vertical gradient $\nabla_{z_g} p_p$ is
approximated at the half pressure level $k+\frac{1}{2}$ by
%
\begin{equation}
  \begin{split}
     \nabla_{z_g} p_{p,k+\frac{1}{2}}  =  10 k_B \biggl[  
   &  \frac{N_{e,k+1}-N_{e,k}}{z_{k+1}-z_{k}} (T_i+T_e)_{k+\frac{1}{2}} + \\
   & \frac{(T_i+T_e)_{k+1} - (T_i+T_e)_{k}}{z_{k+1}-z_{k}} N_{e, k+\frac{1}{2}}  \biggr]
  \end{split}
    \label{eq:gradz_pp}
\end{equation}
%
with the geopotential height $z$ in [cm].
The factor of 10 takes the conversion from [$\frac{g}{s^2 cm^2}$] to  [$\frac{kg}{s^2
m^2}$] into account. \\
%
The plasma pressure gradient in geographic eastward direction at the half pressure 
level $k+\frac{1}{2}$ and geographic longitude $\phi_g$ and geographic latitude
$\lambda_g$ is approximated by
%
\begin{equation}
  \begin{split}
     \nabla_{\phi_g} p_{p,k+\frac{1}{2}}  = & 10 k_B 
        \frac{1}{2 \Delta \phi R_{k+\frac{1}{2}} cos\lambda_g}
         \biggl[  \\
     & \bigl( N_e(\phi_g + \Delta \phi_g ) - N_e(\phi_g -\Delta \phi_g) \bigr)_{k+\frac{1}{2}} 
       (T_i(\phi_g)+T_e(\phi_g))_{k+\frac{1}{2}}  + \\
     & \bigl((T_i+T_e)(\phi_g + \Delta \phi_g )- (T_i+T_e)(\phi_g -\Delta \phi_g) \bigr)_{k+\frac{1}{2}} N(\phi_g)_{e, k+\frac{1}{2}} \biggr]
  \end{split}
  \label{eq:gradphi_pp}
\end{equation}
%
with $\Delta \phi_g$ the discrete step size in the eastward direction and the radius 
$ R_{k+\frac{1}{2}}$ at the half pressure level. \\
%
The plasma pressure gradient in geographic north direction at the half pressure 
level $k+\frac{1}{2}$ and geographic longitude $\phi_g$ and geographic latitude
$\lambda_g$ is approximated by
%
\begin{equation}
  \begin{split}
     \nabla_{\lambda_g} p_{p,k+\frac{1}{2}}  =  & 10 k_B 
        \frac{1}{2 \Delta \lambda R_{k+\frac{1}{2}}}
      \biggl[ \\
      & \bigl( N_e(\lambda_g + \Delta \lambda_g ) - N_e(\lambda_g -\Delta \lambda_g) \bigr)_{k+\frac{1}{2}} 
      (T_i(\lambda_g )+T_e(\lambda_g ))_{k+\frac{1}{2}}  + \\
      & \bigl((T_i+T_e)(\lambda_g + \Delta \lambda_g )- (T_i+T_e)(\lambda_g -\Delta \lambda_g) \bigr)_{k+\frac{1}{2}} 
      N(\lambda_g )_{e, k+\frac{1}{2}} \biggr]
  \end{split}
  \label{eq:gradlam_pp}
\end{equation}
%
with $\Delta \lambda_g$ the discrete step size in the northward direction. At the
geographic poles we set 
%
\begin{equation}
   \nabla_{\lambda_g} p_{p,k+\frac{1}{2}} (\lambda_g= \pm 90^{\circ}) =  0
\end{equation}
%
Inserting the derivatives (\ref{eq:gradz_pp}), (\ref{eq:gradphi_pp}) and
(\ref{eq:gradlam_pp}) into equation (\ref{eq:gradp_p}) lead to
%
\begin{equation}
   \nabla p_p  =  10 k_B \begin{pmatrix} \nabla_{\phi_g} \\
                                           \nabla_{\lambda_g}\\
                                           \nabla_{z_g}
			   \end{pmatrix}
	[(T_i + T_e ) N_e]_{\phi_g, \lambda_g, k+\frac{1}{2}} \label{eq:discgradp_p}
\end{equation}
%
with $\nabla p_p$ in [$\frac{kg}{s^2 m^2}$]. The cross product with the geomagnetic
field vector from equation (\ref{eq:j_p}) can be written as
%
\begin{equation}
  \begin{split}
  \mathbf{J}_{p,\phi_g, \lambda_g, k+\frac{1}{2}} 
       & = -  \frac{10 k_B}{\mathbf{B}_0^2} \biggl(\frac{R}{R_0} \biggr)^3 
                         \begin{pmatrix} \nabla_{\phi_g} \\
                                         \nabla_{\lambda_g}\\
                                         \nabla_{z_g}
			   \end{pmatrix}
	\times \begin{pmatrix}  b_y \\
                                b_x\\
                               -b_z
		\end{pmatrix} 
	[(T_i + T_e ) N_e]_{\phi_g, \lambda_g, k+\frac{1}{2}}  \\
	& =  -  \frac{10 k_B}{\mathbf{B}_0^2} \biggl(\frac{R}{R_0} \biggr)^3 
	 \begin{pmatrix} -\nabla_{\lambda_g} b_{z} - \nabla_{z_g} b_x      \\
                          \nabla_z  b_{y} +\nabla_{\phi_g} b_{z}     \\
                          \nabla_{\phi_g} b_{x}  -  \nabla_{\lambda_g} b_{y}
		\end{pmatrix}		
		   [(T_i + T_e ) N_e]_{\phi_g, \lambda_g, k+\frac{1}{2}} 
  \end{split}
		    \label{eq:currp_p}
\end{equation}
%
The height variation of the magnetic field is approximated by the factor 
$(\frac{R}{R_0})^3$. 
%
The calculated current vectors $\mathbf{J}_g$ and $\mathbf{J}_p$ have to be rotated to 
point in the geomagnetic direction.
%
\begin{gather}
  \frac{1}{D}{{J}}_{e1}^{p,g} = \frac{\mathbf{d}_1}{D} \cdot \mathbf{J}_{pg} \label{eq:rot_j1_pg} \\
  \frac{1}{D}{{J}}_{e2}^{p,g} = \frac{\mathbf{d}_2}{D} \cdot \mathbf{J}_{pg} \label{eq:rot_j2_pg}
\end{gather}
%
with $\mathbf{d}_{1}$ and $\mathbf{d}_{2}$ denoting the vectors in quasi magnetic eastward and down/
equatorward direction. The quantity D varies
with the strength of the geomagnetic field and the distortion of the
geomagnetic field from a pure dipole field. The quantity D can be calculated by
using the vectors $\mathbf{d}_{1}$ and $\mathbf{d}_{2}$
%
\begin{equation}
    D = |\mathbf{d}_1 \times \mathbf{d}_2| \label{eq:D}
\end{equation}
%
The vectors $\mathbf{d}_{1}(h_0)$ and $\mathbf{d}_{2}(h_0)$ as well as $D(h_0)$ are only 
calculated at the reference height $h_0$. The
height variation is approximated by
%
\begin{equation}
    \mathbf{d}_1(h) = \mathbf{d}_{1}(h_0) \biggl( \frac{R_0}{R} \biggr)^{\frac{3}{2}}; \quad\quad
    \mathbf{d}_2(h) = \mathbf{d}_{2}(h_0) \biggl( \frac{R_0}{R} \biggr)^{\frac{3}{2}} 
            \frac{\sqrt{4-3 cos^2 \lambda_m}}{\sqrt{4-3 \frac{R_0}{R} cos^2 \lambda_m}}
\end{equation}
%
Considering equation (\ref{eq:D}) the quantity $\frac{\mathbf{d}_1}{D}$ varies like $\frac{1}{\mathbf{d}_2}$ and 
$\frac{\mathbf{d}_2}{D}$ varies like $\frac{1}{\mathbf{d}_1}$.
%
The current at top pressure level of the model is extrapolated using
%
\begin{gather}
  {J}_{e1,k_{max}+\frac{1}{2}}^{p,g} =\frac{3}{2} {J}_{e1,k_{max}-\frac{1}{2}}^{p,g} - \frac{1}{2}{J}_{e1,k_{max}-\frac{3}{2}}^{p,g} \\
  {J}_{e2,k_{max}+\frac{1}{2}}^{p,g} =\frac{3}{2} {J}_{e2,k_{max}-\frac{1}{2}}^{p,g} - \frac{1}{2}{J}_{e2,k_{max}-\frac{3}{2}}^{p,g}	
\end{gather}
%
with $k_{max}+\frac{1}{2}$ the index of the highest pressure level.
%
\subsection{Field--line Integration}\label{subsubsec:field--line-intg}
%
This section describes how the gravity and plasma pressure gradient driven current
is added to forcing of the electrodynamo equation. The gravity and plasma pressure 
gradient driven current is therefore added to the current driven by the neutral wind.
The current of the different sources are combined
in \texttt{subroutine fieldline\_integrals}, before the field line integration.
The field line integration itself is described in section \ref{cap:fieldlineintg}. \\
%
The height--integrated
eastward current density $K_{m\phi}$, see eq. (\ref{eq:eldy_1}),
is calculated and included on the right hand side
of the electrodynamo equation (\ref{eq:edyn}). The current $J_{e1}^{p,g}$ 
due to gravity and plasma
pressure gradient is added.
%
\begin{equation}
  K_{m\phi} = |sin I_m | \int_{s_L}^{s_U} \frac{{J}_{e1}}{D} ds  = \\
  B_{e3} |sin I_m | \int_{s_L}^{s_U} \left[ \frac{{J}_{e1}^D}{D} +
              \frac{{J}_{e1}^{p,g}}{B_{e3} D}  \right] ds\label{eq:int_kqp}
\end{equation}
%
with s the line--integral variable and ${s_L}$ and ${s_U}$ the lower and upper boundary
of the line--integration. The inclination of the geomagnetic field line at the
reference height $h_0$ on an assumed spherical Earth is $I_m$ and $B_{e3}$ the
component of the geomagnetic field along the field--line. The eastward current
${J}_{e1}$ has a contribution from the neutral winds ${J}_{e1}^D$ and from the
plasma pressure and gravity term ${J}_{e1}^{p,g}$. \\
%
The field--line integration is approximated by a height--integration 
combined with an interpolation
between the height-varying values at the foot point of the field line 
($\lambda_m$) and at the magnetic 
equator ($\lambda = 0$) which is described in section \ref{cap:fieldlineintg}. 
Figure \ref{fig:fieldline_intg} shows schematically that for
a field line with the foot point at the reference height $h_0$ and latitude $\lambda_m$ the
value $x_{fl}$ at height $h$ on the field line is approximated by the value $x_{eq}$ at the equator
and value $x_m$ at foot point $\lambda_m$ and height h. The calculation of the weighting factors $x_{fl}$
and $wgt_{eq}$ for the interpolation are described in section \ref{cap:fieldlineintg} equations 
(\ref{eq:approx_fl}) and (\ref{eq:wgh1}).

Therefore the eastward current  at ($\lambda, h$) on the
field line is approximated by
%
\begin{equation}
 {J}_{e1}^{p,g}(\lambda, h) = wgt_{eq}(h,\lambda_{eq}){J}_{e1,eq}^{p,g}(\lambda_{eq}, h)  + 
            wgt_m(\lambda_m, h) {J}_{e1,m}^{p,g}(\lambda_m, h) \label{eq:approx_fl_j1}
\end{equation}
%
Since the integration is done in height rather than along the field--line,
$ds$ in equation \ref{eq:int_kqp} is expressed in terms of height h.
%
The conversion from field line integration to height is shown in section \ref{cap:fieldlineintg}.
The increment ds along the field line is substituted by eq. (\ref{eq:ds9}).
%
The contribution $K_{m\phi}^{p,g}$ to the height integrated eastward current is the sum over 
all pressure level k
%
\begin{equation}
  \frac{K_{m\phi}^{p,g}}{|sin I_m|} =  
  B_{e3}(h_0) A(h_0) \sum_k  \left[ 
              \frac{10^4 {J}_{e1}^{p,g}(\lambda, h_{k+\frac{1}{2}}) }{B_{e3}(h_0) D(h_{k+\frac{1}{2}})}  \right]  
	     10^{-2} h_{fac}(- d\sqrt{h_A-h_{k+\frac{1}{2}}} )\label{eq:intdes_kqp}
\end{equation}
%
The factor $10^4$ converts the current from [$\frac{A}{cm^2}$] to [$\frac{A}{m^2}$] and
the factor $10^{-2}$ converts ds from [cm] to [m].
%
The north--/upward height integrated current density is
%
\begin{equation}
  K_{m\lambda} = \mp \int_{s_L}^{s_U} \frac{{J}_{e2}}{D} ds  = \\
 \mp  B_{e3}  \int_{s_L}^{s_U} \left[ \frac{{J}_{e2}^D}{D} +
              \frac{{J}_{e2}^{p,g}}{B_{e3} D}  \right] ds\label{eq:int_kql}
\end{equation}
%
With the above mentioned approximation of the field line integration the contribution
from the plasma pressure gradient and the gravity driven current is the sum over 
all pressure level k
%
\begin{equation}
  K_{m\lambda}^{p,g} = \pm  
  B_{e3}(h_0) A(h_0) \sum_k  \left[ 
              \frac{10^4 {J}_{e2}^{p,g}(\lambda, h_{k+\frac{1}{2}}]) }{B_{e3}(h_0) D(h_{k+\frac{1}{2}})}  \right]  
	    10^{-2} h_{fac} (d\sqrt{h_A-h_{k+\frac{1}{2}}} )\label{eq:intdes_kql}
\end{equation}
