%
\section{Solar Input} \label{cap:solar_input}
%
Solar input, ionization rates, dissociation rates, and heating rates, 
are calculated in the module \src{qrj.F,} which contains the \src{subroutines qrj}, 
\src{init\_sflux, init\_qrj, ssflux, and alloc\_q}.  The acronym "qrj" is historically 
based on a convolution of Q (heating rate) and RJ ($O_2$ dissociation rate).
\\
\subsection{Solar Irradiance} \label{cap:solar_irradiance}
%
\subsubsection{The solar irradiance proxy model}
The thermosphere absorbs solar radiation in the soft X-ray ultraviolet 
(XUV, 0.05 nm -- 30 nm), extreme ultraviolet (EUV, 30 nm -- 120 nm), and 
far ultraviolet (FUV, 120 nm -- 200 nm) wavelength ranges, primarily by 
O, $O_2$, and $N_2$ through photon ionization and dissociation. The 
ionization threshold of O, $O_2$, and $N_2$ are 91.3 nm, 102.6 nm, and 
79.8 nm, respectively. The dissociation threshold of $N_2$ is 98.6 nm. 
The spectrum longward of 102.6 nm is mainly absorbed by $O_2$ dissociation,
 especially in the Schumann-Runge continuum from 132--175 nm. \\ 
The TIE-GCM uses the EUVAC solar proxy model as the default solar 
input \cite{Richards1994} in the spectral range from 5--105 nm.  
This model is an empirical representation of solar irradiance and its 
variability. It includes two parts, a reference spectrum at solar 
minimum and a wavelength-dependent solar variability. The variability 
is usually parameterized by solar indices that are historically 
available. The most widely used solar index is the $F_{10.7}$ index. 
Solar flux at a given solar activity level is obtained from $F_{10.7}$ and 
its 81 day average $< F_{10.7} >$ as follows:\\
The EUVAC model \cite{Richards1994} is used between 5 nm and 105 
nm. 
%        
\begin{equation}
   f(\lambda) = f_{ref}(\lambda)[1+A(\lambda)(P-80)]
\end{equation}
%	
where $f_{ref}$ is the reference spectrum, $A$ is the solar variability factor, 
and $P = (F_{10.7} + <F_{10.7}>) / 2.$  It is apparent that the solar spectrum is 
equivalent to the solar minimum reference spectrum when P = 80. For solar 
minimum conditions when P is less than 80, the proxy model imposes the limit 
that solar irradiance is no less than 80\% of the solar minimum reference 
spectrum in any given wavelength band.\\
%
EUVAC is based on the F74113 reference spectrum, with the EUV fluxes between 
15 nm and 25 nm doubled and the EUV flux below 15 nm increased by a factor of 
3; the solar variability scale factors are based on the AE-E and calibration 
rocket measurements. The F74113 spectrum was measured on April 23, 1974 by a 
rocket flight \cite{Heroux1977},\cite{Heroux1978} at low 
solar activity.       EUVAC has been compared with more recent measurements by 
the SEE instrument on the TIMED satellite \cite{Woods2005} and found to be 
in reasonable agreement, although SEE fluxes at low solar activity are slightly 
higher in the XUV wavelengths. \\
%
For the spectral region shortward of 5 nm, GOES X-ray measurements in its two 
channels, 0.05--0.4 nm and 0.1--0.8 nm, are used to establish the reference 
irradiance and the solar variability factor for the first two bins. The third 
bin, i.e., from 0.8--1.8 nm, is based on some early X-ray/XUV measurements as 
described in Solomon and Qian \cite{Solomon2005}.  The Hinteregger model \cite{Hinteregger1981a} 
is used for the wavelength range from 1.8--5 nm, based on the 
SC21REFW reference spectrum and variability factors, but with the fluxes in 
this wavelength range increased by a factor of 3.  The SC21REFW reference 
spectrum is based on the AE-E EUV measurements \cite{Hinteregger1981b}. \\
%
The Woods and Rottman model \cite{Woods2002} is used from 105 nm to 
175 nm. The Woods and Rottman model is based on UARS (Upper Atmosphere Research 
Satellite) SOLSTICE (SOLar STellar Irradiance Comparison Experiment) measurements 
from 119--200 nm and a 1994 rocket measurement.  This model uses a two-parameter 
fit based on $F_{10.7}$ and $<F_{10.7}>$, but the implementation in the TIE-GCM simplifies 
this using the P and A factors as described above for the EUVAC model, which 
results in differences from the original formulation generally less than 1\%. \\
%
Solar input is segmented into optimized low-resolution bands as described by 
Solomon and Qian \cite{Solomon2005}.  This paper includes a detailed description of the 
band structure, the ionization and photoionization scheme, and tables of solar 
flux, cross sections, and branching ratios.  These tables are coded directly 
into the module \index{\src{qrj.F}}.  The reference spectrum and solar variability factors 
for the solar proxy model are given here in table \ref{tab:solar_refspectrum}.  
All values are specified at 1 AU; 
the fluxes are multiplied by the factor $sfeps$ to account for the Earth's 
orbital eccentricity effect on Sun-Earth distance, which is calculated 
in \src{advance.F}\index{\src{advance.F}}.\\

%
\begin{table}[tb]
\begin{tabular}{|c |c|c|c|} \hline
$\lambda_{min}$  & $\lambda_{min}$    &  Reference Spectrum($f_{ref}$) Variability& Factor(A)
\\ \hline \hline
%
    0.05  &    0.40   &       5.010e+01     &       6.240e-01 \\
    0.40  &    0.80   &       1.000e+04     &       3.710e-01 \\
    0.80  &    1.80   &       2.000e+06     &       2.000e-01 \\
    1.80  &    3.20   &       2.850e+07     &       6.247e-02 \\
    3.20  &    7.00   &       5.326e+08     &       1.343e-02 \\
    7.00  &   15.50   &       1.270e+09     &       9.182e-03 \\
   15.50  &   22.40   &       5.612e+09     &       1.433e-02 \\
   22.40  &   29.00   &       4.342e+09     &       2.575e-02 \\
   29.00  &   32.00   &       8.380e+09     &       7.059e-03 \\
   32.00  &   54.00   &       2.861e+09     &       1.458e-02 \\
   54.00  &   65.00   &       4.830e+09     &       5.857e-03 \\
   65.00  &   79.80   &       1.459e+09     &       5.719e-03 \\
   65.00  &   79.80   &       1.142e+09     &       3.680e-03 \\
   79.80  &   91.30   &       2.364e+09     &       5.310e-03 \\
   79.80  &   91.30   &       3.655e+09     &       5.261e-03 \\
   79.80  &   91.30   &       8.448e+08     &       5.437e-03 \\
   91.30  &   97.50   &       3.818e+08     &       4.915e-03 \\
   91.30  &   97.50   &       1.028e+09     &       4.955e-03 \\
   91.30  &   97.50   &       7.156e+08     &       4.422e-03 \\
   97.50  &   98.70   &       4.482e+09     &       3.950e-03 \\
   98.70  &  102.70   &       4.419e+09     &       5.021e-03 \\
  102.70  &  105.00   &       4.235e+09     &       4.825e-03 \\
  105.00  &  110.00   &       3.298e+09     &       3.007e-03 \\
  110.00  &  115.00   &       3.200e+09     &       2.099e-03 \\
  115.00  &  120.00   &       8.399e+09     &       2.541e-03 \\
  121.57  &  121.57   &       3.940e+11     &       4.230e-03 \\
  120.00  &  125.00   &       1.509e+10     &       3.739e-03 \\
  125.00  &  130.00   &       7.790e+09     &       2.610e-03 \\
  130.00  &  135.00   &       2.659e+10     &       2.877e-03 \\
  135.00  &  140.00   &       1.387e+10     &       2.632e-03 \\
  140.00  &  145.00   &       1.824e+10     &       1.873e-03 \\
  145.00  &  150.00   &       2.802e+10     &       1.202e-03 \\
  150.00  &  155.00   &       5.080e+10     &       1.531e-03 \\
  155.00  &  160.00   &       7.260e+10     &       1.125e-03 \\
  160.00  &  165.00   &       1.055e+11     &       1.043e-03 \\
  165.00  &  170.00   &       1.998e+11     &       6.089e-04 \\
  170.00  &  175.00   &       3.397e+11     &       5.937e-04 
\\ \hline \hline
\end{tabular}
\caption{Reference spectrum and solar solar variability factors for 
the solar proxy model used in the TIE-GCM.}
\label{tab:solar_refspectrum}
\end{table}
%
\subsubsection{Use of solar irradiance measurements} 
Although solar proxy models have been widely used in aeronomy studies, 
solar irradiance can deviate significantly from empirical parameterizations,
on time scales ranging from solar flares, to the solar 27-day rotation, to 
the 11-year solar cycle.  Therefore, the TIE-GCM has an option to use measured 
solar irradiance spectra directly.  The TIMED/SEE instrument has measured solar 
spectral irradiance from 0.1--195 nm from 2002 to present \cite{Woods2005}. 
It covers nearly from the solar maximum to the solar minimum of solar cycle 23. 
It uses two types of instruments: the XUV Photometer System (XPS) and the EUV 
Grating Spectrograph (EGS). The XPS measures solar irradiance from 0.1--34 nm 
with a resolution of 5--10 nm. The EGS measures irradiance from 27--195 nm with 
0.4 nm spectral resolution. TIMED/SEE data are available from February 2002 
through March 2011. More information is available at the TIMED/SEE web site: 
http://lasp.colorado.edu/see.  These data are pre-processed from 1-nm resolution 
to the binning scheme in table \ref{tab:solar_refspectrum}, and provided in a netCDF data file for input 
to the TIE-GCM.  If the namelist input parameter \flags{SEE\_NCFILE} is set to specify 
this file, these data will be used for solar irradiance input instead of the 
EUVAC proxy model, interpolated to current model time.  Any environment variables 
imbedded in the file path will be expanded by the model input module. \\

%
\subsubsection{Ionization Rates}
\paragraph{Direct solar ionization rates}
The solar flux in each spectral interval at each level of the atmosphere is 
calculated by applying the Beer-Lambert law:
%        
\begin{equation}
   I(\lambda,z) = I(\lambda,\infty) exp[\tau(\lambda,z)]
\end{equation}
%	
where the optical depth $\tau$ as a function of altitude z is:
%        
\begin{equation}
   \tau(\lambda,z)=\sum_j \sigma_j(\lambda)N_j(z) \cdot Ch
\end{equation}
%		
$N_j$ and $\sigma_j$ are the column density and total absorption cross 
section for each species, and $Ch$ is the Chapman grazing incidence 
integral correction factor, calculated by \src{subroutine chapman}.  The 
process--specific rate for each species j and process k is then:
%        
\begin{equation}
   R_{j,k}(z)=\sum_{\lambda} I(\lambda,z)\sigma_j(\lambda)\beta_{j,k}(\lambda)
\end{equation}
%	
where $\beta_{i,j}$ is a branching ratio, e.g., for ionization, 
dissociative ionization, or dissociation.  Tables containing these 
parameters are available in Solomon and Qian \cite{Solomon2005}, at 
http://download.hao.ucar.edu/pub/stans/euv, and in \src{module qrj.F}. \\

%
\paragraph{Photoelectron ionization rates} 
The photon ionization process generates energetic electrons called 
photoelectrons or secondary electrons. Photoelectrons generated by 
shorter EUV wavelengths have sufficient energy to ionize, dissociate, 
and excite neutral species. The photoelectron flux and its impact on 
ionization, dissociation, and excitation can be calculated using models 
that are based on radiative transfer methods [e.g. \cite{Nagy1970} 
\cite{Solomon1988}]. However, such detailed calculations are not 
practical for a global general circulation model. The method of 
Solomon and Qian \cite{Solomon2005} is used to estimate the additional proportion 
of photoelectron ionization to direct photoionization that occurs in 
each wavelength band, which are also tabulated as specified above.  

\paragraph{Dissociation Rates}
\begin{itemize}
   \item \myemph{EUV and XUV}:
       Dissociation rates of $N_2$ and $O_2$ are obtained from each wavelength 
       band in the EUV and XUV using the tabulated process-specific branching 
       ratios and photoelectron enhancement factors described above.  They 
       are then integrated over wavelength to calculate the production of 
       atoms (and, in the case of dissociative ionization, atomic ions) 
       at each altitude level.
%
    \item \myemph{Lyman-alpha}:
       Ionization of nitric oxide by the solar H Lyman-alpha emission at 
       121.6 nm is calculated using the parameter \flags{beta9}, which is specified 
       as a pseudo rate coefficient in chemrates as the product of the 
       Lyman-alpha solar flux and the NO cross section: 
       %        
       \begin{equation}
         beta9 = 2.91\times 10^{11} \cdot (1+0.2(F_{10.7}-65)/100) \cdot 2\times10^{-18}
       \end{equation}
%
     \item \myemph{Schumann-Runge continuum}:
	Solar flux in the spectral range from 132--175 primarily dissociates 
	$O_2$ in the thermosphere, resulting in production of one ground state 
	$O(^3P)$ atom and one excited $O(^1D)$ atom.  Solar flux in 5 nm bands as a 
	function of altitude is calculated from Beer's law, then multiplied by 
	the $O_2$ dissociation cross section for each band, and integrated across 
	wavelength, to obtain the dissociation rate.
%
     \item \myemph{Schumann-Runge  bands (SRB)}:
	For $O_2$ dissociation and heating in the SR bands (197--200 nm), 
	which results in two ground state $O(^3P)$ atoms, a simple approximation 
	is employed for the optically thin region above ~97 km:
       %   
	\begin{align} 
	  p3f = & \text{(erg/s in SRB)(solar activity factor)(Sun-Earth distance factor)}   \notag \\
	  p3f = & (9.03\times10^{-19}) (1+0.11(F_{10.7}-65.) (\text{sfeps})  \notag \\
          rj  =  &p3f / do2  
	\end{align}
	%
	with do2 is the dissociation energy of $O_2$ in ergs =$8.2\times10^{-12}$.
	
\end{itemize}
%
\paragraph{Heating Rates}
%
\begin{itemize}
	\item \myemph{Neutral heating rates} \\
	\myemph{EUV}:  Direct solar heating of the neutral gas, primarily from photoelectron 
	impact and from the kinetic energy of dissociation products, is approximated 
	as 5\% of the total photon energy absorbed at each altitude level.  This is a 
	small fraction of the heating that is ultimately attributable to ionization 
	and dissociation caused by solar EUV, but these other pathways are calculated 
	as chemical heating, electron heating, and ion heating.\\
	\myemph{FUV}:  The heating rate is calculated by subtracting the sum of the $O_2$ 
	bond energy and the excitation energy of the $O(^1D)$ state (do22=1.1407 erg) 
	from the photon energy, and assuming that the remainder goes into the 
	energy of the products, and ultimately neutral heating.  Quenching of 
	$O(^1D)$ resulting in kinetic and vibrational excitation of $N_2$ and $O_2$ is 
	also assumed to result in neutral heating; this is calculated in \src{module qrj} 
	(rather than in the chemistry module) for historical reasons. \\
	\myemph{SRB}:  33\% of the dissociation rate is assumed to result in neutral 
	heating (approximately the difference between the $O_2$ bond energy and 
	the photon energy in the center of the SRB region).

        \item \myemph{Electron and Ion Heating Rates}\\
	Electron and ion temperatures are calculated in \src{subroutine settei}, 
	see chapter \ref{cap:settei}.

\end{itemize}
