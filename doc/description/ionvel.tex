%
\section{Electrodynamic drift velocity calculation / \index{IONVEL.F}}\label{cap:ionvel}
%
The input to \src{subroutine ionvel} is summarized in table
\ref{tab:input_ionvel}.
%
\begin{table}[tb]
\begin{tabular}{|p{3.5cm} ||c|c|c|c|c|c|} \hline
physical field               & variable        & unit&pressure
level& timestep
\\ \hline \hline
%
electric field: east-,north-, upward&       {}     & $$   &   & \\
geomagnetic direction on geographic grid & $R*E_{m \phi}, R*E_{m
\lambda},2 \Delta z * E_{z}$  & $V$   & interface  & $t_{n}$ \\
geomagnetic field (north-, east-, downward) &       {$B_x, B_y, B_z$} & $Gauss$   & -  & -\\
geomagnetic field strength &       {B}     & $Gauss$   &  - & -\\
Jacobian &       $\mathbf{J}_{mag} = \frac{\partial
s_{mag}}{\partial s_{geo}}$     & -   &  - & -
 \\ \hline
\end{tabular}
\caption{Input fields to \src{subroutine ionvel}}
\label{tab:input_ionvel}
\end{table}
%
The output of \src{subroutine ionvel} is summarized in table
\ref{tab:output_ionvel}.
%
\begin{table}[tb]
\begin{tabular}{|p{3.5cm} ||c|c|c|c|c|c|} \hline
physical field               & variable        & unit&pressure
level& timestep
\\ \hline \hline
%
electric field: geographic east-,north-, upward&       {}     & $$   &   & \\
on geographic grid & $E_{x}, E_{y}, E_{z}$  & $V/cm$   & interface &
        $t_{n}$ \\
electromagnetic drift velocity (geographic east-,north-, upward) &
        $v_{ExB,x}, v_{ExB,y}, v_{ExB,z}$ & cm/s   &  interface & $t_n$
 \\ \hline
\end{tabular}
\caption{Output fields of \src{subroutine ionvel}}
\label{tab:output_ionvel}
\end{table}
%
Firstly, the electric field input to this subroutine $R*E_{m \phi},
R*E_{m \lambda},\Delta z * E_{z}$ which is the electric field on the
geographic grid but in geomagnetic direction is rotated into the
geographic direction by using the Jacobian matrix
$\mathbf{J}_{mag}$.
%
\begin{gather}
   \mathbf{J}_{mag} = \frac{\partial s_{mag}}{\partial s_{geo}} =
   \begin{pmatrix}
      \frac{\cos \lambda_m}{\cos \lambda_g} \frac{d \phi_m}{d \phi_g} & {\cos \lambda_m}\frac{d \phi_m}{d \lambda_g}\\
      \frac{1}{\cos \lambda_g} \frac{d \lambda_m}{d \phi_g}&\frac{\lambda_m}{\lambda_g}
   \end{pmatrix}
\end{gather}
%
with the geographic coordinate system denoted by $s_{geo}$ and the
geomagnetic coordinate system denoted by $s_{mag}$. The Jacobian
matrix is calculated in the \src{subroutine apex} by using the
eastward $\mathbf{f}_1$ and northward $\mathbf{f}_2$ base vector for
quasi--dipole coordinates as defined in \cite{Richmond95}.
%
\begin{gather}
   \mathbf{J}_{mag} =
   \begin{pmatrix}
      \mathbf{f}_{2}(2) & \mathbf{f}_2(1)\\
      -\mathbf{f}_1(2)  & \mathbf{f}_1(1)
   \end{pmatrix}
\end{gather}
%
\emph{check the Jacobian stuff, how does this fit together the two
matrices????????}. \\

%
The electric field [V/cm] in geographic direction is
$\mathbf{E}_{geo} = \mathbf{J}_{mag} \mathbf{E}_{mag}$ ??????? with
$\mathbf{E}_{geo} = (E_x,E_y,E_z)^T$.
%
\begin{align}
     E_x  = & \frac{1}{R}\left[ j_{11} R* E_{m \phi} + j_{21} R* E_{m \lambda} \right]\\
     E_y  = & \frac{1}{R}\left[ j_{12} R* E_{m \phi} + j_{22} R* E_{m \lambda} \right]\\
     E_z  = & \frac{2 \Delta z * E_{z}}{2 \Delta z}
\end{align}
%
with the radius $R = R_E + z$, and $2 \Delta z = z_{k+1}-z_{k-1}$.
The electric field is extrapolated at the lower and upper boundary
with the height index $k_{bot}$ and $k_{top}$.
%
\begin{align}
     E_z(z_{bot})    = & 2 E_z(z_{bot}+\Delta z)-E_z(z_{bot}+2 \Delta z) \\
     E_z(z_{top})    = & 2 E_z(z_{top}-\Delta z)-E_z(z_{top}-2 \Delta z)
\end{align}
%
The height $z_{bot}$ corresponds to $k_{bot}$, and $z_{top}$ to
$k_{top}$ with $z_{bot}+ \Delta z $ is $k_{bot}+ 1$, and $z_{top}-
\Delta z $ is $k_{top}- 1$ etc. The electrodynamic drift velocity is
determined by
%
\begin{align}
     \mathbf{v}_{ExB} = - \frac{\mathbf{E}\times \mathbf{B}}{B^2}
\end{align}
%
The components of $\mathbf{B}$ are $B_x, B_y, \text{and} B_z$ in
northward, eastward and downward direction. To be consistent with
the geographic direction the magnetic field vector reads $(B_y, B_x,
- B_z)^T$.
%
\begin{align}
     v_{ExB,x}  = & - \frac{E_y B_z + E_z B_x}{B^2}\\
     v_{ExB,y}  = &   \frac{E_z B_y + E_x B_z}{B^2}\\
     v_{ExB,z}  = &   \frac{E_x B_x - E_y B_y}{B^2}\\
\end{align}
%
In the code a factor of $10^6$ is used to convert from $[\frac{V}{cm
Gauss}]$ to $[\frac{V}{m T}] = [m/s]$. However, after the periodic
points are set the electrodynamic drift velocity is converted back
from m/s to $[cm/s]$
