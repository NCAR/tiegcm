\section{Solving for the electric potential} \label{cap:solve}
%
In TIEGCM a multigrid solver is used which was developed by
J. Adams. The source code and information can be found at
http://www.scd.ucar.edu\-/css\-/software\-/mudpack/. In the original mudpack solver 
the finite differencing and the setting of all the boundary conditions
are part of solver package. For TIEGCM the stencil and the boundary
conditions  are
already set up in the TIEGCM code, and therefore the multigrid
solver was modified accordingly. We have three different solver
options in TIEGCM which are all based on the mudpack multigrid package.
%
\begin{itemize}
  \item \flags{isolve=0} multigrid solver \src{mudcr2} (see section \ref{cap:mudsol})
  \item \flags{isolve=1} multigrid solver \src{muhcr2} as direct solver (see section 
         \ref{cap:directsol})
  \item \flags{isolve=2} modified multigrid solver \src{mudcr2} (see section 
         \ref{cap:modmudsol})
\end{itemize}
%
%
\subsection{Multigrid solver \texttt{mudcr2}}\label{cap:mudsol}
%
The \texttt{subroutine mud2cr} attempts to compute
the second-order difference approximation to a two-dimensional
linear nonseparable elliptic partial differential equation with cross
derivative terms on a rectangle.  The approximation is generated on a
uniform grid covering the rectangle. The parameters used for the multigrid solver 
in TIEGCM are the following:
%
\begin{itemize}
  \item \flags{intl=0}: this is an input to the subroutine and is zero for the initial
           call to \texttt{mud2cr} to check for errors. After the initial 
	   call of \texttt{mud2cr} \flags{intl=1} will be set and the PDE is solved.
  \item boundary conditions: with x being the longitudinal direction and y the
          latitudinal one from the equator to the pole.
  \begin{itemize}  
      \item \flags{nxa =0 }: flags boundary conditions on the edge x=xa. nxa = 0
    		means the boundary condition is periodic in x on [xa,xb].
      \item \flags{nxb =0}: flags boundary conditions on the edge x=xb.  nxb = 0
    		means the boundary condition is periodic in x on [xa,xb].
      \item \flags{nyc =2}: flags boundary conditions on the edge y=yc (equator). nyc = 2 means
    		that there are mixed derivative boundary conditions at y=yc.
      \item \flags{nyd =1}: flags boundary conditions on the edge y=yd (pole). nyd=1 means
      that the boundary condition is specified and input thru the variable \src{phi}.
  \end{itemize}
  \item  defining the number
    	    of grid points in x (longitude) and y (latitude) direction
  \begin{itemize} 
    \item \flags{ixp = 5}: ixp+1
  	    is the number of points on the coarsest x grid visited during
  	    multigrid cycling.
    \item \flags{jyq = 3}: jyq+1 
  	    is the number of points on the coarsest y grid visited during
  	    multigrid cycling.
    \item \flags{iex = 5}: integer exponent of 2 used in defining the number
    	    of grid points in the x direction 
    \item \flags{jey = 5}: integer exponent of 2 used in defining the number
    	    of grid points in the y direction 
    \item \flags{nx}: number of equally spaced grid points in the interval [xa,xb]
    		$ nx = ixp \; 2^{iex-1} + 1$
    \item \flags{ny}: number of equally spaced grid points in the interval [yc,yd]
    		$ ny = jyq \; 2^{jey-1} + 1$
  \end{itemize}
  \item \flags{iguess = 0:} no initial guess is used  forcing full multigrid cycling
  \item \flags{tolmax = 0.01}: 
        tolmax is the maximum relative error tolerance
	  used to terminate the relaxation iterations. Assume $\Phi_1$
	  and $\Phi_2$ are the last two computed approximations at 
	  the finest grid level. If we define \\
	     $ \Phi_{diff} = max(|\Phi_2(i,j)-\Phi_1(i,j)|) \quad \text{for all} \quad i,j$
	  and \\
	      $ \Phi_{max} = max(|\Phi_2(i,j)|) \quad \text{for all} \quad i,j $
	  then convergence is considered to have occurred if and only if
	      $ \frac{\Phi_{diff}}{\Phi_{max}} < tolmax  $
  \item \flags{maxcy = 150}: if $tolmax > 0.0$
	  is input, which means error control is used, then maxcy is the limit on the number
	  of cycles between the finest and coarsest grid levels. 
	  When the multigrid iteration is working
          correctly only a few cycles are required for convergence.
  \item \flags{method = 3}: method of relaxation. If neither fx (the longitudinal
       second order derivative $\Sigma_{\phi \phi}/ \Delta \phi^2$) or fy (the latitudinal
       second order derivative $\Sigma_{\lambda \lambda}/ \Delta \lambda_0^2$) dominates 
       over the solution region and they
       both vary considerably choose method = 3, which uses line 
       relaxation in both the x and y direction
  \item \flags{nwork}: length of work array \\
        $length = [7(nx+2)(ny+2)+4(11+isx+jsy)nx*ny]/3$
  \item \flags{mgopt}: multigrid options the default values 
     (2,2,1,3) in the vector \src{mgopt} were chosen for
     robustness.
     \begin{itemize}
         \item \flags{mgopt(1) = 2}: w cycling 
         \item \flags{mgopt(2) = 2}: the number of pre--relaxation 
	   sweeps executed before the
           residual is restricted and cycling is invoked at the next
           coarser grid level
         \item \flags{mgopt(3) = 1}: the number of post--relaxation sweeps executed after the cycling
           has been invoked at the next coarser grid level and the residual
           correction has been transferred back
         \item \flags{mgopt(4) = 3}: multicubic prolongation 
	      (interpolation) is used to
              transfer residual corrections and the PDE approximation
              from the coarse to the fine grid within full multigrid cycling.
     \end{itemize}
  \item output \flags{$\Phi$}: solution of PDE which is the electric potential
  \item output \flags{ierror}: indicates invalid input arguments when
          returned positive and nonfatal warnings when returned
          negative.
%
\end{itemize}
%
If no convergence is reached with this version of the multigrid solver the direct
solver described in section \ref{cap:directsol} is used.
%
\subsection{Multigrid solver \src{muhcr2} as direct solver}\label{cap:directsol}
%
This solver which is in \texttt{subroutine muh2cr} is originally a hybrid 
multigrid/direct method which approximates the
same 2-d nonseparable elliptic PDE as the mudpack solver \src{mud2cr}.
Using a direct method combines the efficiency of multigrid iteration 
with the certainty of
a direct method.  The basic algorithm is modified by using banded
Gaussian elimination in place of relaxation whenever the coarsest
subgrid is encountered within the multigrid cycling.
The solver becomes a full direct method if grid size arguments are chosen
so that the coarsest and finest grids coincide, i.e.  choosing iex=jey=1
and ixp=nx-1, jyq=ny-1. This will set the Gaussian elimination
on the finest grid.  In this case, \texttt{subroutine muh2cr} produces a 
direct solution
to the same nonseparable elliptic PDE. In TIEGCM we are using the
solver \src{muhcr2} only as a direct solver.
%
\subsection{Modified multigrid solver \src{mudcr2}}\label{cap:modmudsol}
%
This solver is the same as the solver in section \ref{cap:mudsol} only with the
exception that the residual is calculated with the coefficient
stencil without upwinding (see section \ref{chap:finitediff}). Therefore 
the solution converges toward 
the solution of the direct solver using the 
coefficient stencil without the upwinding method.
In general upwinding introduces unwanted numerical dissipation. 
The unmodified
coefficient stencil is stored in the array \src{cofum}. 
In comparison with the solver in
section \ref{cap:mudsol} the number of relaxation
steps has to be increased to get convergence.
%
\begin{itemize}
    \item \flags{mgopt(2) = 3}: the number of pre--relaxation 
      sweeps executed before the
      residual is restricted and cycling is invoked at the next
      coarser grid level
    \item \flags{mgopt(3) = 2}: the number of post--relaxation sweeps executed after cycling
      has been invoked at the next coarser grid level and the residual
      correction has been transferred back
\end{itemize}
%
All the other parameters remain the same and can be found in section 
\ref{cap:mudsol}.
If no convergence is reached with this multigrid solver the direct
solver described in section \ref{cap:directsol} is called.
%
