%
\section{Joule heating calculation}
%\section{Joule heating calculation  QJOULE.F}\label{capqjoule}
%
For the Joule heating calculation we assume that the energy transfer
$\textbf{J} \cdot \textbf{E}$ is almost equal to $\textbf{J}_{\bot}
\cdot \textbf{E}$ since the electric field along the geomagnetic
field $\textbf{B}$ is much smaller than the perpendicular component.
$\textbf{J}_{\bot}$ is the current component perpendicular to the
geomagnetic field. We neglect the work done by the Amp\`{e}re Force,
and focus on the Joule heating term $\textbf{J}_{\bot} \cdot
\textbf{E}'$, with $\textbf{E}'$ the electric field in the frame of
the neutral wind $\textbf{v}_n$. We can substitute
%
\begin{equation}
 \textbf{J}_{\bot} \cdot \textbf{E}' = \sigma_P \textbf{E}'^2
\end{equation}
%
with $\sigma_P$ denoting the Pederson conductivity. The square of
the electric field can be written as
%
\begin{equation}
  \textbf{E}'^2 = B^2(\frac{\textbf{E} \times \textbf{B}}{B^2}-
  \textbf{v}_{n\bot})^2
\end{equation}
%
with the electrodynamic drift velocity $\textbf{v}_{ExB}$ which is
$\frac{\textbf{E} \times \textbf{B}}{B^2}$, and $textbf{v}_{n,\bot}$
is the neutral component perpendicular to the geomagnetic field. In
the code we substitute $\sigma_P B^2$ by $\lambda_1 = \frac{\sigma_P
B^2}{\rho}$ with $\rho$ the atmospheric mass density, therefore we
calculate
%
\begin{equation}
  Q_J = \frac{\textbf{J}_{\bot} \cdot \textbf{E}'}{\rho} =
  \lambda_1 (\textbf{v}_{ExB}- \textbf{v}_{n\bot})^2
\end{equation}
%
with $Q_J$ the energy rate per mass [erg/s/g] from Joule heating. \\

The Joule heating is added to the equation to solve for the ion and
neutral temperature, see chapter \ref{cap:dt} and \ref{cap:settei}.
\\

In the heat equation for the neutrals there is no explicit term for
the energy transfer from the ion to the neutrals, since we assume
that eventually the whole Joule heating will go into the neutral
temperature.
%
\begin{equation}
  Q_J^{T_n} =
  \lambda_1 (\textbf{v}_{ExB}- \textbf{v}_{n\bot})^2
\end{equation}
%
We refer to \cite{Schunk00} eq. 5.48d, which is part of the
transport equation for the high latitude E-region
%
\begin{equation}
  0 = \sum_n \frac{\nu_{in}}{m_i + m_n} [3k(T_n - T_i) + m_n (\textbf{v}_i-\textbf{v}_n)^2]
\end{equation}
%
??? check with book what the values are Sum over what n???? This
equation assumes that we have monotonic particles, such that we only
have kinetic, and no rotational, energy. In the above equation the
collision frequency $\nu_{in}$ instead of the ion drag coefficient
is used, since it also uses the difference in velocity between the
ion and neutrals $\textbf{v}_i-\textbf{v}_n$, instead of only the
$\textbf{E} \times
\textbf{B}$ drift velocity. \\

According to \cite{Schunk00} only part of the Joule heating
contributes to the ion temperature
%
\begin{equation}
  Q_J^{T_i} = \frac{\overline{m}_n}{\overline{m}_n + \overline{m}_i}
  \lambda_1 (\textbf{v}_{ExB}- \textbf{v}_{n\bot})^2
\end{equation}
%
This approximation is valid above approx. 100 km where the
contributions from the electron neutral collisions to the Pedersen
conductivity $\sigma_P$ is small. Only around 75 km we would have to
consider also the Joule heating part which goes into the electron
temperature. However, since the contribution from the electrons is
small, we can justify using $\lambda_1$ which is based on the
Pedersen conductivity including the electron neutral collisions.
%
\subsection{Joule heating calculation for ion temperature}
%
The input to \src{subroutine qjoule\_ti} is summarized in table
\ref{tab:input_qjoule_ti}.
%
\begin{table}[tb]
\begin{tabular}{|p{3.5cm} ||c|c|c|c|c|c|} \hline
physical field               & variable        & unit&pressure
level& timestep
\\ \hline \hline
%
neutral temperature         & $T_n$           & $K$   & midpoint  & $t_n$\\
mean molecular mass         & $\overline{m}$  & $\frac{g}{mole}$ &interface & $t_{n+1}$ \\
mass mixing ration of $O_2$  & $\Psi_{O_2}$        & mmr  & midpoint & $t_n$\\
number density of ions       &  $N_{O_2^+}$      & $1/cm^3$  & midpoint & $t_n$\\
  {}       &  $N_{O^+}$      & $1/cm^3$  & midpoint & $t_n$ \\
 {}        &  $N_{NO^+}$     & $1/cm^3$  & midpoint & $t_n$ \\
 {}        &  $N_{N^+}$      & $1/cm^3$  & midpoint & $t_n$ \\
 ion drag coefficient (mag.east.)        &  $\lambda_1$      & $1/s$  & interface & $t_n$ \\
 $\textbf{E}\times\textbf{B}$ velocity   &  $\textbf{v}_{ExB}$& $cm/s$ & interface & $t_n$ \\
 dimensionless vertical velocity         &  $W$               & $1/s$  & interface & $t_{n}$ \\
 neutral velocity        &  $\textbf{v}_{n}$     & $cm/s$  & midpoints & $t_n$
 \\ \hline
\end{tabular}
\caption{Input fields to \src{subroutine qjoule\_ti}}
\label{tab:input_qjoule_ti}
\end{table}
%
The output of \src{subroutine qjoule\_ti} is summarized in table
\ref{tab:output_qjoule_ti}.
%
\begin{table}[tb]
\begin{tabular}{|p{3.5cm} ||c|c|c|c|c|c|} \hline
physical field               & variable        & unit&pressure
level& timestep
\\ \hline \hline
%
Joule heating of ion temperature & $Q_J^{T_i}$     & $\frac{ergs}{s
g}$   & midpoints   & $t_n$ $t_{n+1}$
 \\ \hline
\end{tabular}
\caption{Output fields of \src{subroutine qjoule\_ti}}
\label{tab:output_qjoule_ti}
\end{table}
%
\begin{table}[tb]
\begin{tabular}{|p{3.5cm} ||c|c|c|c|c|c|} \hline
variable               & physical name        & value \\ \hline
\hline
%
$N_A$   &  Avogadro number         & $6.023 \times 10^{23} \#/mol$  \\
$g$     &  gravitational acceleration                   & $cm^2/s$  \\
{}      &  at lower boundary  &   \\
$R^*$   &  gas constant       & $8.314 \times 10^{7} erg/K/mol$  \\
$m_{i}$ &  molecular weight         & $g/mol$ \\
        & species i         &
 \\ \hline
\end{tabular}
\caption{Global parameters used in \src{subroutine qjoule\_ti}}
\label{tab:parameters_qjoule_ti}
\end{table}
%
The mean molecular mass of the ions $\overline{m}_{ion}$ at the
midpoints in [g/mol] is calculated by
%
\begin{equation}
  \overline{m}_{ion}=\frac{N_{O^+} m_O+N_{O_2^+}m_{O_2}+N_{N^+}m_{N^{4s}}+
        N_{N_2^+}m_{N_2}+N_{NO^+}m_{NO}}
  {N_{O^+}+N_{O_2^+}+N_{N^+}+N_{N_2^+}+N_{NO^+}}
\end{equation}
%
and the mean molecular mass of the neutrals $\overline{m}_n$ at the
midpoints in [g/mol] is
%
\begin{equation}
 \overline{m}_{n}(z+\frac{1}{2} \Delta z) = 0.5*(\overline{m}_{n}(z)+\overline{m}_{n}(z+\Delta z)
\end{equation}
%
with $\overline{m}_n$ on the interface level with $z$ correspond to
the height index $k$, $z+\Delta z$ to $k+1$, and on the midpoint
level $z+\frac{1}{2}$ correspond to the height index $k$. The scale
height H [cm] at midpoints is
%
\begin{equation}
  H= \frac{R^* T_n}{\overline{m}_n g}
\end{equation}
%
since $T_n$ and $\overline{m}_n$ are on the midpoint level. Note
that the gas constant $R^*$ is in units of [erg/K/mol] and the mean
molecular mass in units of [g/mol]. The vertical neutral velocity
$w$ in [cm/s] is
%
\begin{equation}
  w = W H
\end{equation}
%
with the dimensionless vertical velocity $W$. The dimensionless
vertical velocity at the midpoints is determined by averaging the
values at the height level $k$ and $k+1$ to get the midpoint level
$k$, i.e. $W(z+\frac{1}{2}) = 0.5[W(z)+W(z+\Delta z)]$. The
$\textbf{E}\times \textbf{B}$ drift velocity $\textbf{v}_{ExB}$ in
[cm/s] at the midpoints is $\textbf{v}_{ExB}(z+\frac{1}{2}\Delta z)
= 0.5(\textbf{v}_{ExB}(z)+\textbf{v}_{ExB}(z+\Delta z))$, and the
ion drag coefficient in magnetic eastward direction $\lambda_1$ in
[1/s] is averaged in the same way to get the midpoint values. \\

The Joule heating in [ergs/s/g] which goes into the ion temperature
is
%
\begin{equation}
  Q_J^{T_i}=
  \frac{\overline{m}_n}{\overline{m}_n+\overline{m}_{ion}}
  \lambda_1 (\textbf{v}_{ExB}-\textbf{v}_n)^2
\end{equation}
%
In \src{subroutine qjoule\_tn} the Joule heating for the neutral
temperature is calculated. Since the calculation is the same as for
the Joule heating for the ion temperature we refer to the previous
subsection. The Joule heating for the neutral temperature doesn't
have the ratio of the mean molecular mass of ion to ion plus
neutral. Since we have no explicit heating term for the energy
transfer from ions to neutrals.
%
\begin{equation}
  Q_J^{T_n}=
  \lambda_1 (\textbf{v}_{ExB}-\textbf{v}_n)^2
\end{equation}
%
