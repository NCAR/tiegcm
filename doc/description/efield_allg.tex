%
\section{Electric field calculation / \index{EFIELD.F}}\label{cap:efield}
%
The input to \src{subroutine efield} is summarized in table
\ref{tab:input_efield}.
%
\begin{table}[tb]
\begin{tabular}{|p{3.5cm} ||c|c|c|c|c|c|} \hline
physical field               & variable        & unit&pressure
level& timestep
\\ \hline \hline
%
electric potential on geomag. grid          & $\Phi$           &
$V$   & interfaces  & $t_n$ \\
apex factors & $\frac{\cos \lambda_0}{\cos \lambda_m^*}$& -&- &-  \\
              & $\frac{ \partial \lambda_0}{\partial \lambda_m^*}$& -&-&-
 \\ \hline
\end{tabular}
\caption{Input fields to \src{subroutine efield}}
\label{tab:input_efield}
\end{table}
%
The output of \src{subroutine efield} is summarized in table
\ref{tab:output_efield}.
%
\begin{table}[tb]
\begin{tabular}{|p{3.5cm} ||c|c|c|c|c|c|} \hline
physical field               & variable        & unit&pressure
level& timestep
\\ \hline \hline
%
electric field: east-,north-, upward&       {}     & $$   &   & \\
geomagnetic direction on geographic grid & $R*E_{m \phi}, R*E_{m
\lambda},\Delta z * E_{z}$  & $V$   & interface  & $t_{n}$
 \\ \hline
\end{tabular}
\caption{Output fields of \src{subroutine efield}}
\label{tab:output_efield}
\end{table}
%
The electric field calculation was started in the \src{subroutine
efield} and is finished in the \src{subroutine ionvel}. Note that
the electric field is also calculated in the dynamo-module in
\src{subroutine threed}, however it's not used. Later versions of
TIE-GCM will use the electric field calculated in the dynamo-module,
and therefore the \src{subroutine efield} will be shortened. \\

Input to the \src{subroutine efield} is the three dimensional
electric potential $\Phi$ on the geomagnetic grid. It's copied into
a local array to set up the wrap around points. The three
dimensional electric field in magnetic eastward, northward and
upward direction $E_{m \phi}$, $E_{m \lambda}$ and $E_z$ is
%
\begin{align}
   E_{m \phi}    = & - \frac{1}{R cos \lambda_m} \frac{\partial \Phi}{\partial \phi_m}\\
   E_{m \lambda} = & \frac{1}{R } \frac{\partial \Phi}{\partial
   \lambda_m} \\
   E_{z}  = & \frac{\partial \Phi}{\partial z}
\end{align}
%
with the geomagnetic coordinates $\phi_m$ and $\lambda_m$ for
longitude and latitude. $R$ is the radius $R=R_E+z$. In the code the
equally spaced grid points distribution $\lambda_0$ in $\lambda_m^*$
is used to calculate the derivatives. Therefore the mapping factors
from the irregular latitudinal spaced grid $\lambda_m^*$ to
$\lambda_0$ have to be taken into account. This leads to the
discrete derivatives
%
\begin{align}
   R E_{m \phi}(\phi_m,\lambda_m,z)    = & - \frac{cos \lambda_0}{ cos \lambda_m^*}
           \frac{\Phi(\phi_m+\Delta \phi,\lambda_m,z)-\Phi(\phi_m-\Delta \phi,\lambda_m,z)}{2 cos \lambda_0\Delta \phi_m}\\
   R E_{m \lambda}(\phi_m,\lambda_m,z) = & \frac{\partial \lambda_0}{\lambda_m^* }
   \frac{ \Phi(\phi_m,\lambda_m+\Delta \lambda,z)-\Phi(\phi_m,\lambda_m-\Delta \lambda,z)}{2 \Delta \lambda_0} \\
   2\Delta z E_{z}(\phi_m,\lambda_m,z)  = & - \Phi(\phi_m,\lambda_m,z+\Delta z)+\Phi(\phi_m,\lambda_m,z-\Delta z)
\end{align}
%
The factor $\frac{\cos \lambda_0}{ \cos \lambda_m^*}$ is denoted by
\src{rcos0s} in the source code and $\frac{\partial
\lambda_0}{\partial \lambda_m^* }$ by \src{dt0dts}, which are
calculated in the apex-module. The polar values of $R E_{m \phi}$
are set by taking the value of $R E_{m \lambda}$ which is shifted by
180 degree. \\
%
The electric field  values $R E_{m \phi}$ , $R E_{m \lambda}$ and
$\Delta z E_{z}$ are then mapped from the geomagnetic grid to the
geographic grid. These values are then further used in the
\src{subroutine ionvel} and rotated in the geographic direction.
